\section{Normal Extensions}

\begin{defn}
    An algebraic extension $\E/\F$ is called a \deff{normal extension} if whenever $f(x) \in \F[x]$ is irreducible and has a root in $\E$, then $f(x)$ splits into linear factors in $\E[x]$.
\end{defn}

\begin{defn}
    Let $\E/\F$ be an extension and let $\mathcal{F} = \left\{ f_i(x) \right\}_{i \in \mathcal{I}}$ be a (possibly infinite) family of non-constant polynomials in $\F[x]$. Then, $\E$ is said to be a \deff{splitting field for the family $\mathcal{F}$ over $\F$} if each $f_i(x) \in \mathcal{F}$ splits as a product of linear factors in $\E[x]$ and is generated by the roots of the polynomials. 
\end{defn}

\begin{rem} \label{rem:splitting-field-for-family-exists}
    A splitting field for any family always exists since an algebraic closure exists. So, we consider $A \subseteq \overline{\F}$ to be the set of roots of all polynomials in the family, and put $\E \vcentcolon= \F(A) \subseteq \overline{\F}$.
\end{rem}

\begin{prop}
    Let $\F$ be a field and let $\mathcal{F} \subseteq \F[x]$ be a family of separable polynomials. Then, $\E/\F$ is separable where $\E \subseteq \overline{\F}$ is the splitting field of $\mathcal{F}$ over $\F$.
\end{prop}

\begin{proof}
    Let $a \in \E = \F(A)$ where $A$ is as in \Cref{rem:splitting-field-for-family-exists}. By \Cref{cor:exists-a-finite-set-with-a-in-F(B)}, there is a finite set $\{a_1, \ldots, a_n\} \subseteq A$ such that $a \in \F(a_1, \ldots, a_n)$. Since each $a_i$ is a root of a separable polynomial, it is separable. Applying \Cref{cor:separable-element-generates-separable-extension} repeatedly, we see that $\F(a_1,\ldots,a_n)/\F$ is a separable extension, hence $a$ is separable over $\F$.
\end{proof}

\begin{lem} \label{lem:F-embedding-is-automorphism}
    Let $\E/\F$ be an algebraic extension. Let $\sigma \colon \E \to \E$ be an $\F$-embedding. Then, $\sigma$ is an automorphism of $\E$.
\end{lem}
\begin{proof}
    We only need to prove that $\sigma$ is onto. Let $\alpha \in \E$ be arbitrary. Put $p(x) \vcentcolon= \irr(\alpha, \F)$. Let $\K \subseteq \E$ be the subfield of $\E$ generated by the roots of $p(x)$ in $\E$. Then, $\K$ is a finite dimensional vector space over $\F$ and $\alpha \in \K$. Since $\sigma$ is an $\F$-embedding, it maps roots of $p(x)$ to roots of $p(x)$. Thus, $\sigma(\K) \subseteq \K$. Since $\sigma$ is an $\F$-linear map and $\K$ is a finite dimensional vector space over $\F$, $\sigma\restr{\K}$ is onto and contains $\alpha$ in its image. Since $\alpha \in \E$ was arbitrary, we are done.
\end{proof}

\begin{theorem} \label{thm:normality-automorphic-embedding}
    Let $\F$ be a field and fix an algebraic closure $\overline{\F}$ of $\F$. Let $\F \subseteq \E \subseteq \overline{\F}$ be fields. Then, the following are equivalent. 
    \begin{enumerate}
        \item Every $\F$-embedding $\sigma \colon \E \to \overline{\F}$ is an automorphism of $\E$.
        \item $\E$ is a splitting field of a family of polynomials in $\F[x]$.
        \item $\E/\F$ is a normal extension.
    \end{enumerate}
\end{theorem}
\begin{proof}
    ($1 \implies 2$) Let $a \in \E$ and $p_a(x) = \irr(a,\F)$. If $b \in \overline{\F}$ is a root of $p_a(x)$, then there exists an $\F$-isomorphism $\F(a) to \overline{\F}$ with $a \mapsto b$. Extend this to a map $\sigma \colon \E \to \overline{\F}$. By hypothesis, we have $\E = \sigma(\E) \ni b$. Thus, $\E$ is a splitting field of the family $\left\{ p_a(x) \right\}_{a \in \E}$.
    
    ($2 \implies 3$) Let $\E$ be a splitting field of the family $\left\{ p_i(x) \right\}_{i \in I} \subseteq \F[x]$ over $\F$. Let $f(x) \in \F[x]$ be an irreducible having a root $a \in \E$. Let $b \in \overline{\F}$ be any root of $f(x)$. There exists an $\F$-embedding $\F(a) \to \overline{\F}$ with $a \mapsto b$. Extend this to an $\F$-embedding $\sigma \colon \E \to \overline{\F}$. Since $\sigma$ fixes $\F$, it maps roots of $p_i(x)$ to its roots for all $i \in I$. Since $\E$ is generated by these roots, we see that $\sigma(\E) \subseteq \E$ and hence $b \in \E$. Thus, all roots of $f(x)$ lie in $\E$, and hence, $f(x)$ splits linearly over $\E$. Since $f(x)$ was an arbitrary irreducible polynomial, we are done.
    
    ($3 \implies 1$) Let $\sigma \colon \E \to \overline{\F}$ be an $\F$-embedding. Let $a \in \E$. Then, $p(x) \vcentcolon= \irr(a, \F)$ splits linearly over $\E$. Since $\sigma$ fixes $\F$, $\sigma(a)$ is a root of $p(x)$, and thus $\sigma(a) \in \E$. Thus, $\sigma(\E) \subseteq \E$. By \Cref{lem:F-embedding-is-automorphism}, we get that $\sigma$ is an automorphism of $\E$. (Note that $\E/\F$ is indeed algebraic since $\E \subseteq \overline{\F}$.)
\end{proof}

\begin{prop}
    Let $\F \subseteq \E_1, \E_2 \subseteq \K$ be fields. Suppose that $\E_i/\F$ are normal. Then, so are $\E_1\E_2/\F$ and $(\E_1 \cap \E_2) / \F$.
\end{prop}
\begin{proof}
    Fix an algebraic closure $\overline{\F} \supseteq \K$. Let $\sigma \colon \E_1 \E_2 \to \overline{\F}$ be an $\F$-embedding. Then, $\sigma(\E_1 \E_2) = \sigma(\E_1) \sigma(\E_2) = \E_1 \E_2$. Since this is true for any $\F$-embedding, $\E_1\E_2/\F$ is normal by \Cref{thm:normality-automorphic-embedding}. Similar calculations also hold for the intersection.
\end{proof}

\begin{ex}
    Quadratic extensions are always normal. Pick $\alpha \in \E \setminus \F$, then $\E = \F(\alpha)$ is a splitting field of $\irr(\alpha, \F)$ over $\F$.
\end{ex}

\begin{rem}
    Unlike the ``tower laws'' for algebraic and separable extensions, the ``composition'' of normal extensions need not be normal. For example, consider the chain
    \[
        \Q \subseteq \Q(\sqrt{2}) \subseteq \Q(\sqrt[4]{2}).
    \]
    Each successive extension is quadratic and hence normal. However, $\Q(\sqrt[4]{2}) / \Q$ is not normal since the irreducible (via \nameref{prop:ES-criterion}) polynomial $x^4 - 2 \in \Q[x]$ has a root in $\Q(\sqrt[4]{2})$ but does not factor completely. However, one part of the ``tower law'' does hold, as can be easily verified. 
\end{rem}

\begin{prop}
    Let $\F \subseteq \E \subseteq \K$ be fields. If $\K/\F$ is normal, then so is $\K/\E$.
\end{prop}
