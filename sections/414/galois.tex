\section{Galois Extensions}

\subsection{Introduction}

\begin{defn}
    A field extension $\E/\F$ is called a \deff{Galois extension} if it is normal and separable. The \deff{Galois group of a Galois extension} $\E/\F$ is the group of all $\F$-automorphisms of $\E$ under composition. It is denoted as $\Gal(\E/\F)$.
    
    If $f(x) \in \F[x]$ is a separable polynomial and $\E$ is a splitting field of $f(x)$ over $\F$, then $\E/\F$ is a Galois extension and the \deff{Galois group of $f(x)$ over $\F$} is defined to be $\Gal(\E/\F)$ and is denoted as $\Gal(f(x), \F)$ or simply $\G_f$ if $\F$ is clear from context.
\end{defn}

\begin{rem}
    The definition of $\Gal(f(x), \F)$ does not depend on the splitting field chosen, up to isomorphism. Let $\E$ and $\E^{\prime}$ be two splitting fields of $f(x)$ over $\F$. By the \nameref{thm:iso-SF}, there is an $\F$-isomorphism $\tau \colon \E \to \E^{\prime}$. Then, $\sigma \mapsto \tau \circ \sigma \circ \tau^{-1}$ is an isomorphism from $\Gal(\E/\F)$ to $\Gal(\E^{\prime}/\F)$.
\end{rem}

\begin{ex}
    \phantom{hi}
    \begin{enumerate}
        \item Let $\E/\F$ be an extension of finite fields. Then, $\abs{F} = q$ and $\abs{\E} = q^n$ for some prime power $q$ and some $n \in \N^+$. Then, $\E$ is a splitting field for $x^{q^n} - x \in \F[x]$ over $\F$. Thus, the extension is normal. Since the fields are finite, it is also separable. Thus, $\E/\F$ is Galois. 
        
        \item The extension $\Q(\sqrt[3]{2}) / \Q$ is \textbf{not} Galois. Since $\ch(\Q) = 0$, it is separable by \Cref{cor:zero-char-perfect}. However, it is not normal since the irreducible (via \nameref{prop:ES-criterion}) polynomial $x^3 - 2 \in \Q[x]$ has a root in $\Q(\sqrt[3]{2})$ but does not split as a product of linear factors.
    \end{enumerate}
\end{ex}

\begin{prop} \label{prop:Galois-order-is-degree}
    Let $\E/\F$ be a finite Galois extension. Then, $\abs{\Gal(\E/\F)} = \left[ \E\colon\F \right]_s = \left[ \E\colon\F \right]$. 
\end{prop}

\begin{proof}
    Fix an algebraic closure $\overline{\F} \supseteq \E$. Let $n \vcentcolon \left[ \E\colon\F \right]_s$. Let $\sigma_1, \ldots, \sigma_n \colon \E \to \overline{\F}$ be $\F$-embeddings. Since $\E/\F$ is normal, $\sigma_i \in \Gal(\E/\F)$. Thus, $\abs{\Gal(\E/\F)} \geq n$. On the other hand, if $\sigma \in \Gal(\E/\F)$, then $\sigma$ is an $\F$-embedding of $\E$ into $\overline{\F}$ when composed with the inclusion. Thus, $\Gal(\E/\F) = \left\{ \sigma_1, \ldots, \sigma_n \right\}$. The last equality follows by \Cref{thm:separable-iff-degrees-equal}.
\end{proof}

\begin{rem}
    The above proposition shows why both normality and separability are needed. If the extension is normal but not separable, then the order of the group would be the separable degree, which would not be equal to the degree by \Cref{thm:separable-iff-degrees-equal} again.
    
    On the other hand, if the extension was separable but not normal, then there would be an extension $\sigma \colon \E \to \overline{\F}$ that would map $\E$ outside $\E$ and so, not all extensions will belong to the Galois group. 
    
    For example, consider $\Q(\sqrt[3]{2})/\Q$. Since there is only one root of $x^3 - 2 \in \Q[x]$ in $\Q(\sqrt[3]{2})$, there is only one $\Q$-automorphism of $\Q(\sqrt[3]{2})$.
\end{rem}

\begin{prop}
    Let $q$ be a prime power. The Galois group of the Galois extension $\F_{q^n} / \F_q$ is the cyclic group of order $n$ generated by the Frobenius automorphism $\varphi \colon \F_{q^n} \to \F_{q^n}$ defined by $a \mapsto a^q$.
\end{prop}
\begin{proof}
    $\varphi$ is an $\F_q$-automorphism since any $a \in \F_q$ satisfies $x^q - x$. Thus, $\varphi \in \Gal(\F_{q^n}/\F_q)$. By \Cref{prop:Galois-order-is-degree}, we have $\abs{\Gal(\F_{q^n}/\F_q)} = n$. Thus, it suffices to show that $\varphi$ has order at least $n$. Let the order of $\varphi$ be $d$. Then, $d \leq n$. Note that
    \[
        \varphi^d(a) = a^{q^d}.
    \]
    If $\varphi^d = \id_{\F_{q^n}}$, then every $a \in F_{q^n}$ satisfies $x^{q^d} - x$. Thus, $q^d \geq q^n$ and thus $d \geq n$. Hence, $d = n$. 
\end{proof}

\begin{ex}
    An extension $\K/\F$ is called \deff{biquadratic} if $\left[ \K \colon \F \right] = 4$ and $\K$ is generated over $\F$ by roots of two irreducible separable quadratic polynomials. 
    In particular, $\K/\F$ is Galois. Write $\K = \F(\alpha, \beta)$ and let $p(x) \vcentcolon= \irr(\alpha, \F)$ and $q(x) \vcentcolon= \irr(\beta,\F)$. Let $\overline{\alpha}, \overline{\beta} \in \K$ denote the other root of $p(x)$ and $q(x)$ respectively. By separability, $\alpha \neq \overline{\alpha}$ and $\beta \neq \overline{\beta}$.
    
    Since $\left[ \F(\alpha,\beta) \colon \F \right] = 4$, $p(x)$ is irreducible over $\F(\beta)$ and $q(x)$ over $\F(\alpha)$. The four automorphisms are thus determined by sending $\alpha$ to $\alpha$ or $\overline{\alpha}$ and $\beta$ to $\beta$ or $\overline{\beta}$.
    
    Define the automorphisms $\tau, \sigma \colon \K \to \K$ by
    \begin{align*}
        \tau(\alpha) = \overline{\alpha}&, \tau(\beta) = \beta, \\
        \sigma(\alpha) = \alpha &, \tau(\beta) = \overline{\beta}.
    \end{align*}
    Then, $\tau^2 = \sigma^2 = \id_{\K}$. Thus, $\Gal(\K/\F) = \Z_2 \times \Z_2$, the Klein-four group, $V_4$. 
\end{ex}

\begin{ex}[Galois group of a separable cubic]
    Let $\F$ be a field with $\ch(\F) \neq 2,3$. Let $f(x) = x^2 + px + q \in \F[x]$ be an irreducible cubic. In particular, $f(x)$ has no roots in $\F[x]$. Note that 
    \[
        f^{\prime}(x) = 3x^2 + p \neq 0
    \]
    since $\ch(\F) \neq 3$. Thus, $f(x)$ is separable by \Cref{prop:irreducible-separable-zero-char}. Thus, a splitting field $\E$ of $f(x)$ over $\F$ must have degree either $3$ or $6$. Thus, by \Cref{prop:Galois-order-is-degree}, $\abs{\Gal(\E/\F)} = 3$ or $6$. We now show how the discriminant determines this. 
    
    Let $\E = \F(\alpha_1, \alpha_2, \alpha_3)$, where $f(x) = \prod_{i=1}^3 (x-\alpha_i)$. Any $\sigma \in \Gal(\E/\F)$ permutes these roots. Let $p_{\sigma} \in S_3$ denote the corresponding permutation. Then, $\sigma \mapsto p_{\sigma}$ is injective. Under this, we identify $\Gal(\E/\F)$ with a subgroup of $S_3$. Thus, $\Gal(\E/\F) = A_3$ or $S_3$. 
    
    Let 
    \[
        \delta = (\alpha_1 - \alpha_2) (\alpha_2 - \alpha_3) (\alpha_3 - \alpha_1).
    \]
    Then, $\delta^2 = \disc(f(x)) = -(4p^3 + 7q^2) \in \F$. Thus, $\left[ \F(\delta) \colon \F \right] \leq 2$. Now, if $\delta \in \F$, then $\Gal(\E/\F)$ cannot have any odd permutations since they do not fix $\delta$, and thus $\Gal(\E/\F) = A_3$. On the other hand, if $\delta \notin \F$, then $2 = \left[ \F(\delta) \colon \F \right] \divides \left[ \E \colon \F \right]$, and thus $\Gal(\E/\F) = S_3$.  
    
    Note that $\delta \in \F \iff \disc(f(x))$ is a perfect square in $\F$, so the above is completely determined by the discriminant being a perfect square. For example, if $f(x) = x^3 + x + 1 \in \Q[x]$, then $\disc(f(x)) = -31$ and $\Gal(\E/\Q) \cong S_3$. On the other hand, if $f(x) = x^3 + 3x + 1$, then $\disc(f(x)) = 81 = 9^2$, and thus, $\Gal(\E/\Q) \cong A_3$.
\end{ex}

\subsection{The Fundamental Theorem of Galois Theory}

\begin{defn}
    Let $\E$ be a field and let $G$ be \emph{a} group of automorphisms of $\E$. Then, 
    \[
        \E^G \vcentcolon= \left\{ a \in \E \mid \sigma(a) = a \text{ for all } \sigma \in G \right\}
    \]
    is called the \deff{fixed field of $G$ acting on $\E$}.
\end{defn}

\begin{theorem}[Fundamental Theorem of Galois Theory (FTGT)]
    Let $\K/\F$ be a \emph{finite} Galois extension. Consider the sets
    \[
        \mathcal{I} = \left\{ \E \mid \E \text{ is an intermediate field of $\K/\F$} \right\} \text{ and } \mathcal{G} = \left\{ H \mid H \leq \Gal(\K/\F) \right\}.
    \]
    \begin{enumerate}
        \item The maps 
        \[
            \E \mapsto \Gal(\K/\E) \text{ and } H \mapsto \K^H
        \]
        gives a one-to-one correspondence between $\mathcal{I}$ and $\mathcal{G}$, called the \deff{Galois correspondence}. Moreover, this correspondence is inclusion reversing.
        
        \item $\E/\F$ is Galois iff $\Gal(\K/\E) \trianglelefteq \Gal(\K/\F)$, and in this case
        \[
            \Gal(\E/\F) \cong \frac{\Gal(\K/\F)}{\Gal(\K/\E)}.
        \]
        
        \item $\K/\E$ is always Galois and $\abs{\Gal(\K/\E)} = \left[ \K \colon \E \right] = \dfrac{\left[ \K \colon \E \right]}{\left[ \E \colon \F \right]}$.
        
        \item If $\E_1, \E_2 \in \mathcal{I}$ correspond to $H_1, H_2 \in \mathcal{G}$, then $\E_1 \cap \E_2$ corresponds to $ \left\langle H_1, H_2 \right\rangle$ and $\E_1\E_2$ corresponds to $H_1 \cap H_2$.
    \end{enumerate}
\end{theorem}
\begin{proof}
    Omitted due to time constraints.
\end{proof}