\section{Algebraic Closure of a Field}

\begin{defn}
    A field $\K$ is called an \deff{algebraically closed field} if every non-constant polynomial $f(x) \in \K[x]$ has a root in $\K$.
\end{defn}
\begin{defn}
    Let $\K/\F$ be a field extension. We say that $\K$ is an \deff{algebraic closure of $\F$} if $\K$ is algebraically closed and $\K/\F$ is an algebraic extension.
\end{defn}

We have the following simple proposition.

\begin{prop}
    \phantom{hi}
    \begin{enumerate}
        \item $\K$ is algebraically closed iff every non-constant polynomial in $\K[x]$ factors as a product of linear factors. 
        \item $\C$ is algebraically closed.
        \item If $\K$ is algebraically closed and $\bbL/\K$ is an algebraic extension, then $\bbL = \K$.
    \end{enumerate}
\end{prop}
\begin{prop} \label{prop:algebraic-closure}
    Let $\K/\F$ be an extension where $\K$ is algebraically closed. Define,
    \[
        \A \vcentcolon= \left\{ \alpha \in \K \mid \alpha \text{ is algebraic over } \F \right\}.
    \]
    Then, $\A$ is an algebraic closure of $\F$.
\end{prop}
\begin{proof}
    By \Cref{cor:A/F-is-algebraic}, we already know that $\A/\F$ is an algebraic extension. Hence, it remains to show that $\A$ is algebraically closed. Let $f(x) \in \A[x]$ be non-constant. Then, $f(x)$ has a root $\alpha \in\K$ since $\K$ is algebraically closed. Thus. $\alpha$ is algebraic over $\A$, and hence over $\F$, by \Cref{cor:algebraic-transitive}. Thus, $\alpha \in \A$.
\end{proof}

\begin{lem} \label{lem:sequence-of-fields}
    Let $\{\F_i\}_{i \geq 1}$ be a sequence of fields with
    \[
        \F_1 \subseteq \F_2 \subseteq \cdots
    \]
    and let $\F \vcentcolon= \bigcup_{i \geq 1} \F_i$. Then, $\F$ is a field with the following operations: Given $a,b \in \F$, there exist smallest $i,j \in \N$ such that $a \in \F_i$ and $b \in \F_j$. Then, $a,b \in \F_{i+j}$ and we define $a+b$ and $ab$ to be the corresponding elements from $\F_{i+j}$.
    
    Moreover, each $\F_i$ is a subfield of $\F$.
\end{lem}
\begin{proof}
    We leave this as an exercise to the reader. Note that we have used ``smallest'' just to ensure that the operations are well-defined. Of course, since $\F_i \subseteq \F_j$ (by which we always mean that $\F_i$ is a subfield of $\F_j$) for any $i \leq j$, we may pick any $i$ and $j$.
\end{proof}

\begin{theorem}[Existence of Algebraically Closed Extension] \label{thm:existence-of-ACE}
    Let $\F$ be a field. Then, there exists an algebraically closed field containing $\F$.
\end{theorem}
\begin{proof}[Proof. (Artin)]
    We first show that given any field $\F$, we can construct a field $\F_1 \supseteq \F$ containing roots of any non-constant polynomial in $\F[x]$. Let $S$ be a set of indeterminates which are in one-to-one correspondence with the set of non-constant polynomials in $\F[x]$. Let $x_f \in S$ denote the indeterminate corresponding to $f$. 
    
    Consider the polynomial ring $\F[S]$. Let
    \[
        I \vcentcolon= \left\langle f(x_f) \mid f \in \F[x], \deg f \geq 1 \right\rangle
    \]
    be the ideal generated by the polynomials $f(x_f) \in S$. We now show that $1 \notin I$, so that $I$ is a proper ideal of $\F[S]$. Suppose that $1 \in I$. Then, 
    \[
        1 = g_1 f_1(x_{f_1}) + \cdots + g_n f_n(x_{f_n})
    \]
    for some $g_1, \ldots, g_n \in \F[S]$. Note that the polynomials $g_j$ involve only finitely many variables. Let $x_i \vcentcolon= x_{f_i}$ for $i=1,\ldots,n$ and let $x_{n+1},\ldots,x_m$ be the remaining variables in $g_1,\ldots,g_n$. Then, we have
    \[
        \sum_{i=1}^n \, g_i\left( x_1, \ldots, x_n, x_{n+1}, \ldots, x_m \right) f_i(x_i) = 1.
    \]
    Now, let $\E \supseteq \F$ be an extension containing roots $\alpha_i$ of $f_i$ ($\E$ exists thanks to \Cref{thm:extension-field-that-contains-root}). Then, putting $x_i = \alpha_i$ for $i=1,\ldots,n$ and $x_{n+1}=\cdots=x_m = 0$, we arrive at a contradiction.
    
    Hence, $I$ is a proper ideal of $\F[S]$, and is thus contained in some maximal ideal $\mathfrak{m} \subseteq \F[S]$. Put $\F_1 \vcentcolon= \F[S]/\mathfrak{m}$. Then, $\F_1$ is a field extension of $\F$. Moreover, $\overline{x_f} \vcentcolon= x_f + \mathfrak{m} \in \F_1$ is a root of $f(x) \in \F[x]$. Thus, $\F_1$ is an extension of $\F$ in which every non-constant polynomial of $\F[x]$ has a root. 
    
    Repeating this procedure, we get a sequence of fields
    \[
        \F =\vcentcolon \F_0 \subseteq \F_1 \subseteq \F_2 \cdots
    \]
    such that every non-constant polynomial in $\F_i$ has a root in $\F_{i+1}$.
    
    Now, put $\K = \bigcup_{i \geq 0} \, \F_i$. $\K$ is a field by \Cref{lem:sequence-of-fields} that has each $\F_i$ as a subfield. Now, if $f(x) \in \K[x]$, then $f(x) \in \F_n[x]$ for some $n$. Thus, $f(x)$ has a root in $\F_{n+1}[x] \subseteq \K$, as desired.
\end{proof}

\begin{cor}[Existence of Algebraic Closure] \label{cor:existence-of-AC}
    Every field $\F$ has an algebraic closure.
\end{cor}
\begin{proof}
    By \Cref{thm:existence-of-ACE}, there exists an algebraically closed field $\bbL \supseteq \F$. Now, use \Cref{prop:algebraic-closure}.
\end{proof}

\begin{prop} \label{prop:embedding-extension}
    Let $\sigma \colon \F \to \bbL$ be an embedding of fields and let $\bbL$ be algebraically closed. Let $\alpha \in \K \supseteq \F$ be algebraic over $\F$ and let $p(x) = \irr(\alpha, \F)$. Suppose $p(x) = \sum a_i x^i$ and define $p^{\sigma}(x) \vcentcolon= \sum \sigma(a_i) x^i$. Then, $\tau \mapsto \tau(\alpha)$ is a bijection between the following sets. 
    \[
        \left\{ \tau \colon \F(\alpha) \to \bbL \mid \tau \text{ is an embedding and } \tau\restr{\F} = \sigma \right\} \leftrightarrow \left\{ \beta \in \bbL \mid p^{\sigma}(\beta) = 0 \right\}.
    \]
\end{prop}
\begin{proof}
    First, we note that the map is indeed well-defined. Let $\tau$ be an embedding that extends $\sigma$. Then, 
    \[
        \tau(p(\alpha)) = p^{\sigma}(\tau(\alpha)) = 0.
    \]
    Thus, $\tau(\alpha)$ is indeed a root of $p^{\sigma}$.
    
    Now, let $\beta \in \bbL$ be such that $p^{\sigma}(\beta) = 0$. Define $\tau_{\beta} \colon \F(\alpha) \to \bbL$ by $\tau_{\beta}(f(\alpha)) = f^{\sigma}(\beta)$ for $f(x) \in \F[x]$. We show that $\tau_{\beta}$ is well-defined. Suppose $f(\alpha) = g(\alpha)$. Then, $(f-g)(\alpha) = 0$, so that $p(x) \divides f(x) - g(x)$ by \Cref{prop:irreducible-monic-poly}. Hence, $p^{\sigma}(x) \divides f^{\sigma}(x) - g^{\sigma}(x)$, giving us $f^{\sigma}(\beta) = g^{\sigma}(\beta)$. Hence, $\tau_{\beta}$ is well-defined. Moreover, it is clearly a homomorphism that extends $\sigma$. It is easily seen that $\beta \mapsto \tau_{\beta}$ is a two-sided inverse of the map $\tau \mapsto \tau_{\alpha}$.
\end{proof}

\begin{rem}
    The above proposition essentially says that the number of ways to extend from $\F$ to $\F(\alpha)$ is precisely the number of roots that $p^{\sigma}(x)$ has in $\bbL$. In particular, this set is non-empty since $\bbL$ is algebraically closed. 
\end{rem}

\begin{theorem} \label{thm:isomorphic-embedding-extension}
    Let $\sigma \colon \F \to \bbL$ be an embedding where $\bbL$ is algebraically closed. If $\K/\F$ is an algebraic extension, then there exists an embedding $\tau \colon \K \to \bbL$ that extends $\sigma$. 
    
    Moreover, if $\K$ is an algebraic closure of $\F$, and $\bbL$ is an algebraic closure of $\sigma(\F)$, then $\tau$ is an isomorphism extending $\sigma$. 
\end{theorem}
\begin{proof}
    Consider the set
    \[
        \Sigma \vcentcolon= \left\{ (\E, \tau) \mid \F \subseteq \E \subseteq \K \text{ and } \tau \colon \E \to \bbL \text{ such that } \tau\restr{\F} = \sigma \right\}.
    \]
    Note that $\Sigma$ is non-empty since $(\F,\sigma) \in\Sigma$. Define the relation $\leq$ on $\Sigma$ by
    \[
        (\E, \tau) \leq (\E^{\prime}, \tau^{\prime}) \iff \E \subseteq \E^{\prime} \text{ and } \tau^{\prime}\restr{\E} = \tau.
    \]
    Then, $(\Sigma, \leq)$ is a partially ordered set. Moreover, if $\Lambda = \left\{ (\E_i, \tau_i) \right\}_{i \in I}$ is a chain in $\Sigma$, then $\E \vcentcolon= \bigcup_{i \in I} \, \E_i$ is a subfield of $\K$, and $\tau \colon \E \to \bbL$ defined as $\tau(x) \vcentcolon= \tau_i(x)$ for $x \in \E_i$ is well-defined. Furthermore, $(\E, \tau)$ is an upper bound on $\Lambda$. By Zorn's Lemma, there exists a maximal element $(\E, \tau) \in \Sigma$. We show that $\E = \K$. If not, pick $\alpha \in \K \setminus \E$. By \Cref{prop:embedding-extension}, we can extend $\tau$ to an embedding $\tau^{\prime} \colon \E(\alpha) \colon \bbL$, which contradicts the maximality of $(\E,\tau)$. Thus, $\tau \colon \K \to \bbL$ is the desired embedding that extends $\sigma$.
    
    Suppose now that $\K$ is an algebraic closure of $\F$ and $\bbL$ of $\sigma(\F)$. We have
    \[
        \sigma(\F) \subseteq \tau(\K) \subseteq \bbL.
    \]
    Thus, $\bbL / \tau(\K)$ is also algebraic. Since $\tau(\K)$ is algebraically closed, we get $\bbL = \tau(\K)$, which proves that $\tau$ is surjective, and hence an isomorphism. (Since $\tau$ is an embedding, it is injective to begin with).
\end{proof}

\begin{cor}[Isomorphism of Algebraic Closures] \label{cor:iso-AC}
    If $\K_1$ and $\K_2$ are two algebraic closures of $\F$, then they are $\F$-isomorphic.
\end{cor}
\begin{proof}
    Consider the inclusion map $i \colon \F \xhookrightarrow{} \K_2$. \Cref{thm:isomorphic-embedding-extension} allows us to extend it to an $\F$-isomorphism $\tau \colon \K_1 \to \K_2$.
\end{proof}

\begin{defn}
    Given a field $\F$, we use $\overline{\F}$ to denote an algebraic closure of $\F$.
\end{defn}

\begin{theorem}[Isomorphism of Splitting Fields] \label{thm:iso-SF}
    Any two splitting fields $\E$ and $\E^{\prime}$ of a non-constant polynomial $f(x) \in \F[x]$ over $\F$ are $\F$-isomorphic.
\end{theorem}
\begin{proof}
    Let $\overline{\E}$ be an algebraic closure of $\E$. Then, it is also one of $\F$. Thus, there exists an embedding $\tau \colon \E^{\prime} \to \overline{\E}$ that extends the inclusion $i \colon \F \xhookrightarrow{} \overline{\E}$, by \Cref{thm:isomorphic-embedding-extension}. 
    
    Let $f(x) = a(x-\alpha_1)\cdots(x-\alpha_n)$ be a factorisation of $f(x)$ in $\E^{\prime}[x]$. Then,
    \[
        f^{\tau}(x) = a(x-\tau(\alpha_1)) \cdots (x-\tau(\alpha_n)) \in \overline{\E}[x].
    \]
    We have $\E^{\prime} = \F(\alpha_1,\ldots,\alpha_n)$ and so $\tau(\E^{\prime}) = \F(\tau(\alpha_1), \ldots, \tau(\alpha_n))$. Thus, $\tau(\E^{\prime})$ is a splitting field of $f^{\tau}$ over $\F$. But $f^{\tau} = f$ since $\tau$ extends the inclusion map. Thus, $\tau(\E^{\prime}) = \E$ since any algebraic closure contains a unique splitting field. 
\end{proof}