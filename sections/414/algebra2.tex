\newpage
\section{Algebraic Extensions}

\subsection{The Prime Subfield}

\begin{defn}
    A \deff{number field} is a subfield of $\C$.
\end{defn}
\begin{prop}
    Any number field contains the field $\Q$.
\end{prop}
\begin{proof}
    We leave this as a simple exercise to the reader. 
\end{proof}

\begin{defn}
    The \deff{characteristic} of a field $\F$ is the smallest positive integer $p$ such that $p \cdot 1 = 0$ if such a $p$ exists, and is defined to be $0$ otherwise. Here, $p \cdot 1$ is defined as
    \[
        p\cdot1 \vcentcolon= \underbrace{1 + \cdots + 1}_{p\text{ times}}.
    \]We denote the characteristic of $\F$ as $\ch(\F)$.
\end{defn}
\begin{prop} \label{prop:char-basics}
    Let $\F$ be a field. Then, the following are true.
    \begin{enumerate}
        \item $\ch(\F)$ is either $0$ or a prime.
        \item If $\ch(\F) = p$, a prime and if $n \cdot 1 = 0$ for some $n \in \Z$\footnotemark, then $p \divides n$.
        \item If $\ch(\F) = p$, a prime, then for any $\alpha \in \F$,
        \[  
            p \cdot \alpha = \underbrace{\alpha + \cdots + \alpha}_{p\text{ times}} = 0.
        \]
        \item If $\ch(\F) = p$, a prime, then for any $x,y \in \F$, 
        \[
            (x+y)^p = x^p + y^p.
        \]
    \end{enumerate}
\end{prop}
\footnotetext{We define $(-n)\cdot1 \vcentcolon= -(n\cdot1)$ for positive $n$, and define $0\cdot1 \vcentcolon= 0$.}
\begin{proof}
\phantom{hi}
\begin{enumerate}
    \item Observe that 
    \[
        m \cdot 1 + n \cdot 1 = (m + n) \cdot 1 \text{ and}
    \]
    \[
        (m \cdot 1)(n \cdot 1) = (mn) \cdot 1
    \]
    for $m,n \in \N^+$. It follows that $\ch(\F)$ is either $0$ or prime. Suppose that $\ch(\F)$ is some composite number $n=ab$ ($a,b \in \N^+$ and $a,b < n$). We then have $n \cdot 1 = 0 \implies (ab) \cdot 1 = 0 \implies (a \cdot 1)(b \cdot 1) = 0$. Since $\F$ is a field, it follows that one of $a\cdot 1$ or $b\cdot 1$ must be zero, which is a contradiction since $a,b < n$.
    
    \item This follows trivially from the first part itself, along with an elementary application of the \nameref{prop:div_algo}.
    
    \item This is trivial as well, since $p \cdot \alpha = p \cdot (1\alpha) = (p\cdot 1)\alpha = 0$. 
    
    \item This is again trivial and follows from the the third part. (Hint: Binomial Theorem)
\end{enumerate}
\end{proof}

\begin{cor} \label{cor:p-power-of-sum}
    Let $\F$ be a field of characteristic $p > 0$. Then, for any $a_1,\ldots,a_n \in \F$, we have
    \[
        (a_1 + \cdots + a_n)^p = a_1^p + \cdots + a_n^p.
    \]
\end{cor}
\begin{proof}
    Apply induction on part $4$ of \Cref{prop:char-basics}.
\end{proof}
\begin{defn}
    Let $\F$ be a field. The \deff{prime subfield} of $\F$ is the subfield of $\F$ generated by the multiplicative identity, $1 \in \F$.
\end{defn}

\begin{prop}
    Let $\F$ be a field. If $\ch(\F) = 0$, then the prime subfield of $\F$ is isomorphic to $\Q$. If $\ch(\F) = p$ for some prime $p$, then the prime subfield of $\F$ is isomorphic to $\Z_p =\vcentcolon \F_p$.
\end{prop}

\begin{proof}
    We have the natural ring homomorphism $\varphi \colon \Z \to \F$ defined by
    \[
        \varphi(n) = n \cdot 1.
    \]
    Note that $\ker\varphi = (\ch(\F) \Z$. Depending on the characteristic, quotienting by the kernel gives us an injection of either $\Z$ or $\Z_p$ into $\F$. Since $\F$ is a field, the \nameref{thm:universal-property} tells us that $\F$ must either contain an isomorphic copy of $\Q$, the field of fractions of $\Z$ (when $\ch(\F) = 0$), or an isomorphic copy of $\F_p \vcentcolon= \Z_p$, the field of fractions of $\Z_p$ (when $\ch(\F) = p$).
\end{proof}

\subsection{Extensions and Degrees}

\begin{defn}
    Let $\F$ be a subfield of $\K$. We say that $\K$ is an \deff{extension field} of $\F$ and we call $\F$ the \deff{base field}. We denote this as $\K/\F$.
\end{defn}
\begin{rem}
    Note that $\K/\F$ is not a quotient. In fact, since the only ideals of $\K$ are $0$ and $\K$, quotienting does not make sense in the first place. 
\end{rem}
\begin{defn}
    Let $\K/\F$ be a field extension. We may regard $\K$ as a vector space over $\F$. We denote $\dim_{\F}\K$ as $[\K \colon \F]$ and call it the \deff{degree} of the field extension $\K/\F$.
\end{defn}
\begin{defn}
    A field extension $\K/\F$ is said to be a \deff{finite extension} if $[\K \colon \F]$ is finite.
\end{defn}
\begin{defn}
    A field extension $\K/\F$ is said to be a \deff{simple extension} if there exists $\alpha \in \K$ such that $\K = \F(\alpha)$.
\end{defn}

\begin{defn}
    Let $\K/\F$ be a field extension and let $\alpha \in \K$. $\alpha$ is said to be \deff{algebraic over $\F$} if there exists a non-zero polynomial $f(x) \in \F[x]$ such that $f(\alpha) = 0$. 
    
    $\alpha$ is called \deff{transcendental over $\F$} if it is not algebraic over $\F$.
    
    If every element of $\K$ is algebraic over $\F$, then $\K/\F$ is called an \deff{algebraic extension}.
\end{defn}

\begin{prop}
    Let $\F \subseteq \mathbb{E} \subseteq \K$ be fields and let $\alpha \in \K$. If $\alpha$ is algebraic over $\F$, then $\alpha$ is algebraic over $\mathbb{E}$. 
\end{prop}
\begin{proof}
    We leave this as an exercise to the reader.
\end{proof}
\begin{cor}
    Let $\F \subseteq \mathbb{E} \subseteq \K$ be fields. If $\K/\F$ is algebraic, then so are $\K/\mathbb{E}$ and $\mathbb{E}/\F$.
\end{cor}

\begin{prop} \label{prop:finite-is-algebraic}
    Every finite extension is an algebraic extension.
\end{prop}
\begin{proof}
    Let $\K/\F$ be a finite extension and let $n \vcentcolon= \dim_{\F}\K$. Let $a \in \K$ be an arbitrary element. We show that $\alpha$ is algebraic over $\F$. Since the set $\{1, \alpha, \ldots, \alpha^n\}$ has $n+1$ elements, it is linearly dependent over $\F$. Thus, there $b_0, \ldots, b_n \in \F$ not all $0$ such that
    \[
        b_0 + b_1 a + \cdots + b_n a^n = 0.
    \]
    Thus, $f(x) \vcentcolon= b_0 + b_1 x + \cdots + b_n x^n \in \F[x]$ is a non-zero polynomial satisfying $f(\alpha) = 0$.
\end{proof}

\begin{ex}
    \phantom{hi}
    \begin{enumerate}
        \item Consider the extensions $\Q \subseteq \R \subseteq \C$. It is known that $\pi \in \R$ is transcendental over $\Q$. A consequence of this is that $\pi \iota \in \C$ is also transcendental over $\Q$. However, $\pi \iota$ is algebraic over $\R$ since it satisfies the non-zero polynomial $x^2 + \pi^2 \in \R[x]$. 
        
        
        Thus, the property of being algebraic depends on the base field. In particular, we have shown that $\C/\Q$ is not an algebraic extension, whereas $\C/\R$ is, by \Cref{prop:finite-is-algebraic}, since $[\C \colon \R] = 2$.
        
        \item Let $\K$ be a finite field and let $\F$ be its prime subfield. Then, $\K$ is a finite dimensional vector space over $\F$ (since $\K$ is finite) and thus, $\K/\F$ is an algebraic extension by \Cref{prop:finite-is-algebraic}.
    \end{enumerate}
\end{ex}

\begin{prop} \label{prop:irreducible-monic-poly}
    Let $\K/\F$ be a field extension and let $\alpha \in\K$ be algebraic over $\F$. Then, the following are true (with ``irreducible'' meaning ``irreducible'' over $\F[x]$). 
    \begin{enumerate}
        \item There exists a unique monic irreducible polynomial $f(x) \in \F[x]$ such that $f(\alpha) = 0$.
        \item $f(x)$ generates the kernel of the map $\F[x] \to \F[\alpha] \subseteq \K$ defined by $p(x) \mapsto p(\alpha)$.
        \item If $g(x) \in \F[x]$ is such that $g(\alpha) = 0$, then $f(x) \divides g(x)$.
        \item $f(x)$ has the least positive degree among all polynomials in $\F[x]$ that are satisfied by $\alpha$. 
    \end{enumerate}
\end{prop}
\begin{proof}
    Define $\psi \colon \F[x] \to \K$ by $p(x) \mapsto p(\alpha)$. Since $\alpha$ is algebraic over $\F[x]$, $I \vcentcolon= \ker\psi$ is non-zero. By \Cref{cor:F[x]-is-UFD-and-PID}, $\F[x]$ is a principal ideal domain, and hence $I = \langle f(x) \rangle$ for some $f(x) \in \F[x]$. Moreover, $f(x)$ is non-zero since $I$ is non-zero. By the \nameref{thm:ring-iso-1}, $\F[x]/I$ is isomorphic to a subring of $\K$, and hence is an integral domain. By \Cref{prop:ideal-characterisation-using-quotient-ring}, $I$ is a prime ideal, and hence $f(x)$ is prime. By \Cref{prop:PID-prime-iff-irreducible}, $f(x)$ is irreducible. Scaling by an appropriate factor, we may assume that $f(x)$ is monic. Clearly, any other $g(x)$ that is satisfied by $\alpha$ must be an element of $I$, and thus $f(x) \divides g(x)$. In particular, if $g(x)$ is irreducible and monic, then $f(x) \divides g(x) \implies g(x) = af(x)$ for some $a \in \F^{\times}$. Since $g(x)$ is also monic, we have that $a = 1$, giving us $f(x) = g(x)$. Thus, such an $f(x)$ is unique.
    
    This proves the first three parts. The fourth part follows from the third via a simple application of the \nameref{prop:div_algo_fields}.
\end{proof}

\begin{defn}
    Let $\K/\F$ be a field extension, and let $\alpha \in \K$ be algebraic over $\F$. The unique irreducible monic polynomial in $\F[x]$ that is satisfied by $\alpha$ is called the \deff{minimal polynomial of $\alpha$ over $\F$}. We denote this as $\irr(\alpha, \F)$.
    
    The degree of $\irr(\alpha, \F)$ is called the \deff{degree of $\alpha$ over $\F$} and is denoted as $\deg_{\F}\alpha$.
\end{defn}

\begin{ex}
    Let $\alpha \in \C$ be a square root of $\iota$. Then, $\alpha$ satisfies $f(x) \vcentcolon= x^4 + 1$. We may show that $f(x) = \irr(\alpha, \Q)$, and hence $\deg_{\Q}\alpha = 4$. However, $\alpha$ also satisfies $x^2 - \iota$, so that $\irr(\alpha, \Q(\iota)) = x^2 - \iota$, and $\deg_{\Q(\iota)}\alpha = 2$. Hence, the degree also depends on the base field.
\end{ex}

\begin{prop}
    Let $\K/\F$ be a field extension and $\alpha \in \K$ be algebraic over $\F$. Let $f(x) \vcentcolon= \irr(\alpha, \F)$ and let $n \vcentcolon= \deg f(x)$. Then, 
    \begin{enumerate}
        \item $\F[\alpha] = \F(\alpha) \cong \F[x]/\langle f(x) \rangle$.
        \item $\dim_{\F}(\F(\alpha)) = n$ and $\{1, \alpha, \ldots, \alpha^{n-1}\}$ is an $\F$-basis of $\F(\alpha)$.
    \end{enumerate}
\end{prop}
\begin{proof}
    Consider the substitution homomorphism $\psi \colon \F[x] \to \F[\alpha]$ defined by $p(x) \mapsto p(\alpha)$. By \Cref{prop:irreducible-monic-poly}, $\ker\psi = \langle f(x) \rangle$. By \Cref{cor:F[x]-is-UFD-and-PID}, $\F[x]$ is a principal ideal domain. Hence, $f(x)$ is a prime element by \Cref{prop:PID-prime-iff-irreducible}, and hence $\langle f(x) \rangle$ is a prime ideal. Since $f(x) \neq 0$, we get that $\langle f(x) \rangle$ is also a maximal ideal, by \Cref{prop:prime-is-maximal-PID}. Moreover, $\psi$ is clearly surjective so that $\im\psi = \F[\alpha]$. Hence, by the \nameref{thm:ring-iso-1}, $\F[x]/\langle f(x) \rangle \cong \F[\alpha]$. Since $\langle f(x) \rangle$ is maximal, $\F[x]/\langle f(x) \rangle$ is a field by \Cref{prop:ideal-characterisation-using-quotient-ring}. Thus, $\F[\alpha] = \F(\alpha)$.
    
    Consider the set $B = \{1, \alpha, \ldots, \alpha^{n-1}\}$. Using $f(x)$, we may write all higher powers of $\alpha$ as an $\F$-linear combination of elements of $B$. Hence, $B$ spans $\F[\alpha]$. Now, suppose $a_0, \ldots, a_{n-1} \in \F$ satisfy
    \[
        a_0 + a_1 \alpha + \cdots + a_{n-1} \alpha^{n-1} = 0.
    \]
    Then, $g(x) \vcentcolon= a_0 + a_1 x + \ldots + a_{n-1} x^{n-1} \in \F[x]$ is a polynomial that is satisfied by $\alpha$. However since $\deg g(x) < \deg f(x)$, by \Cref{prop:irreducible-monic-poly}, we get that $g(x) = 0$. This proves linear independence.
\end{proof}
\begin{cor}
    Let $\K/\F$ be a field extension and let $\alpha \in \K$ be algebraic over $\F$. Then, $\F(\alpha)/\F$ is a finite, and hence, algebraic extension.
\end{cor}

\begin{defn}
    Let $\F \subseteq \E_1, \E_2$ be fields. A \deff{$\F$-homomorphism} from $\E_1$ to $\E_2$ is a field homomorphism $\varphi \colon \E_1 \to \E_2$ that fixes $\F$.
    
    If $\varphi$ is an isomorphism, we call it an \deff{$\F$-isomorphism}.
\end{defn}

\begin{prop}
    Let $\K/\F$ be a field extension and let $\alpha, \beta \in \K$ be algebraic over $\F$. Then, the following two statements are equivalent. 
    \begin{enumerate}
        \item There exists an $\F$-isomorphism $\psi \colon \F(\alpha) \to \F(\beta)$ such that $\psi(\alpha) = \beta$.
        \item $\irr(\alpha, \F) = \irr(\beta, \F)$.
    \end{enumerate}
\end{prop}
\begin{proof}
    ($1 \implies 2$) Let $\psi \colon \F(\alpha) \to \F(\beta)$ be as mentioned. Let $f(x) \vcentcolon= \irr(\alpha, \F)$ and $g(x) \vcentcolon= \irr(\beta, \F)$. Then,
    \begin{align*}
        0 &= \psi(0) \\
        &= \psi(f(\alpha)) \\
        &= f(\psi(\alpha)) &\text{(since $\psi$ is an $\F$-isomorphism)} \\
        &= f(\beta).
    \end{align*}
    Thus, by \Cref{prop:irreducible-monic-poly}, $g(x) \divides f(x)$. Since both are irreducible and monic, $g(x) = f(x)$. 
    
    ($2 \implies 1$) Let $f(x) \vcentcolon= \irr(\alpha, \F) = \irr(\beta, \F)$. The isomorphisms $\F(\alpha) \cong \F[x]/\langle f(x) \rangle \cong \F(\beta)$ are $\F$-isomorphisms, and thus, so is their composition.
\end{proof}

\begin{defn}
    A field extension $\K/\F$ is called a \deff{quadratic extension} if $[\K \colon \F] = 2$. 
\end{defn}
\begin{rem}
    Every quadratic extension is simple. If $\K/\F$ is a quadratic extension and $\alpha \in K \setminus \F$, then $[\F(\alpha) \colon \F] > 1$, and thus $[\F(\alpha) \colon \F] = 2$. Thus, $\F(\alpha) = \K$ and $\K/\F$ is simple.
\end{rem}

\begin{defn}
    A chain of fields $\F_1 \subseteq \ldots \subseteq \F_n$ is called a \deff{tower of fields} if $\F_i$ is a subfield of $\F_{i+1}$ for all $i=1,\ldots,n-1$.
\end{defn}
\begin{prop}[Tower Law] \label{prop:tower}
    Let $\F \subseteq \E \subseteq \K$ be a tower of fields. Then, 
    \[
        \left[ \K \colon \F \right] = \left[ \K \colon \E \right] \cdot \left[ \E \colon \F \right].
    \]
    In particular, the left side is $\infty$ iff the right side is.
\end{prop}
\begin{proof}
    If $\K/\F$ is a finite extension, then so are $\K/\E$ (any finite basis for $\K/\F$ is a spanning set for $\K/\E$) and $\E/\F$ ($\E$ is an $\F$-subspace of $\K$). Thus, if either of $\K/\E$ or $\E/\F$ is not a finite extension, then neither is $\K/\F$. 
    
    Now, suppose $[\K\colon\E] =\vcentcolon n$ and $[\E\colon\F] =\vcentcolon m$ are both finite. Let $\{\alpha_i\}_{i=1}^n \subseteq \K$  be an $\E$-basis of $\K$, and let $\{\beta_j\}_{j=1}^m \subseteq \E$ be an $\F$-basis of $\E$. Now, put $B \vcentcolon= \left\{ \alpha_i \beta_j \mid 1 \leq i \leq n, 1 \leq j \leq m \right\}$. We show that $B$ is an $\F$-basis for $\K$. 
    
    Let $a \in \K$ be arbitrary. Then, 
    \[
        a = \sum_{i=1}^n \, a_i \alpha_i
    \]
    for $a_i \in \E$. For each $i =1, \ldots, n$, we may write
    \[
        a_i = \sum_{j=1}^m \, b_{ij} \beta_j
    \]
    for $b_{ij} \in \F$. Now, 
    \[
        a = \sum_{i=1}^n \sum_{j=1}^m \, b_{ij} (\alpha_i \beta_j)
    \]
    is an $\F$-linear combination of elements of $B$. Hence, $B$ spans $\K$.
    
    Now, suppose $\left\{ b_{ij} \mid 1 \leq i \leq n, 1 \leq j \leq m \right\} \subseteq \F$ satisfies 
    \[
        \sum_{i=1}^n \sum_{j=1}^m \, b_{ij} (\alpha_i \beta_j) = 0.
    \]
    We may group the terms to get
    \[
        \sum_{i=1}^n \left[ \sum_{j=1}^m \, b_{ij} \alpha_i \right] \beta_j = 0.
    \]
    Linear independence of $\{\beta_j\}_{j=1}^m$ forces $\sum_{j=1}^m b_{ij} \alpha_i = 0$ for all $i$, Now, linear independence of $\{\alpha_i\}_{i=1}^n$ forces $b_{ij} = 0$ for all $i,j$, which proves linear independence. It remains to show that $\abs{B} = nm$, which we leave as an exercise.
\end{proof}
\begin{cor}
Let $\K/\F$ be a finite extension and let $\alpha \in \K$. Then, $\deg_{\F} \alpha$ divides $[\K \colon \F]$.
\end{cor}
\begin{proof}
    This follows from the \nameref{prop:tower} by considering the tower $\F \subseteq \F(\alpha) \subseteq \K$. Note that since $\K/\F$ is a finite extension, $\alpha$ is algebraic over $\F$.
\end{proof}

\begin{prop} \label{prop:field-generated-by-algebraic-is-finite-extension}
Let $\K/\F$ be a field extension and let $\alpha_1, \ldots, \alpha_n \in \K$ be algebraic over $\F$. Then, $\F(\alpha_1, \ldots, \alpha_n)$ is a finite (and hence, algebraic) extension of $\F$.
\end{prop}
\begin{proof}
    We leave the details of the proof to the reader. The following tower might be helpful. 
    \[
        \F \subseteq \F(\alpha_1) \subseteq \F(\alpha_1, \alpha_2) \cdots \subseteq \F(\alpha_1, \ldots, \alpha_n) \qedhere
    \]
\end{proof}

\begin{cor} \label{cor:algebraic-transitive}
    Let $\E/\F$ and $\K/\E$ be algebraic extensions. Then, $\K/\F$ is an algebraic extension. 
\end{cor}
\begin{proof}
    Let $\alpha \in \K$ and let $\irr(\alpha,\E) = a_0 + a_1x+ \cdots a_{n-1}x^{n-1} + x^n =\vcentcolon f(x)$. Let $\bbL = \F(a_0, \ldots, a_{n-1})$. Then, $\bbL/\F$ is a finite extension since each $a_i \in \E$ is algebraic over $\F$. Moreover, $0 \neq f(x) \in \bbL[x]$. Thus, $\alpha$ is algebraic over $\bbL$ and $\bbL(\alpha)/\bbL$ is a finite extension. By the \nameref{prop:tower}, $\bbL(\alpha)/\F$ is a finite extension. Hence, $\alpha$ is algebraic over $\F$ by \Cref{prop:finite-is-algebraic}. 
\end{proof}

\begin{cor} \label{cor:A/F-is-algebraic}
Let $\K/\F$ be a field extension. Then, 
\[
    \A \vcentcolon= \left\{ \alpha \in \K \mid \alpha \text{ is algebraic over } \F \right\}
\]
is a subfield of $\K$ containing $\F$. Moreover, $\A/\F$ is an algebraic extension.
\end{cor}
\begin{proof}
    It is clear that $\A$ contains $\F$. We now show that $\A$ is a subfield. Let $\alpha, \beta \in \A$ with $\beta \neq 0$. Then, $\bbL \vcentcolon= \F(\alpha, \beta)$ is a finite extension of $\F$. Thus, all elements of $\bbL$ are algebraic over $\F$. In particular, so are $\alpha \pm \beta$, $\alpha \beta$ and $\alpha \beta^{-1}$.
\end{proof}

\subsection{Compositum of Fields}

\begin{defn}
    Let $\E_1,\E_2 \subseteq \K$ be fields. The \deff{compositum} of $\E_1$ and $\E_2$ is the smallest subfield of $\K$ containing $\E_1$ and $\E_2$. We denote this by $\E_1\E_2$.
\end{defn}

\begin{ex}
    \phantom{hi}
    \begin{enumerate}
        \item Suppose $\F \subseteq \E_1, \E_2 \subseteq \K$ and $\E_1 = \F(\alpha_1, \ldots, \alpha_n)$. Then,
        \[
            \E_1\E_2 = \E_2(\alpha_1, \ldots, \alpha_n).
        \]
        
        \item Let $m$ and $n$ be coprime positive integers. Consider the subfields $\F \vcentcolon= \Q(\zeta_m)$ and $\E \vcentcolon= \Q(\zeta_n)$ of $\C$. Then, 
        \[
            \E\F = \Q(\zeta_{mn}).
        \]
        It is clear that $\E\F \subseteq \Q(\zeta_{mn})$, since $\zeta_n = \zeta^m_{mn}$ and $\zeta_m = \zeta^n_{mn}$. On the other hand, since $\gcd(m,n) = 1$, by \nameref{prop:bezout}, there exist integers $a,b \in \Z$ such that $am + bn =1$. Thus, 
        \[
            \frac{a}{n} + \frac{b}{m} = \frac{1}{mn}
        \]
        which gives us $\zeta_{mn} = \zeta_n^a\zeta_m^b$ and thus $\Q(\zeta_{mn}) \subseteq \E\F$.
    \end{enumerate}
\end{ex}

\begin{prop} \label{prop:finite-dimensional-vector-space-implies-ID-is-field}
    Let $\F$ be a field that is a subring of an integral domain $R$. If $R$ is finite dimensional as an $\F$ vector space, then $R$ is a field.
\end{prop}
\begin{proof}
    It suffices to show that every non-zero element in $R$ has a multiplicative inverse in $R$. Let $a \in R$ be arbitrary with $a \neq 0$. Since $\dim_{\F}R < \infty$, there is a smallest $n \geq 1$ such that the set $\{1, a, \ldots, a^n\}$ is linearly dependent. Then, let $b_0, \ldots, b_n \in \F$ be not all zero such that
    \[
        b_0 + b_1 a + \cdots + b_n a^n = 0.
    \]
    If $b_n = 0$, then the minimality of $n$ is contradicted. Hence, $b_n \neq 0$. If $b_0 = 0$, we may cancel $a$ (since $R$ is an integral domain and $a \neq 0$) to again contradict the minimality of $n$. Thus, $b_0 \neq 0$. Now, we have
    \[
        a(b_1 + \cdots b_na^{n-1}) = -b_0
    \]
    which shows that
    \[
        -\frac{1}{b_0} (b_1 + \cdots + b_na^{n-1}) \in R
    \]
    is a multiplicative inverse of $a$.
\end{proof}

\begin{prop}
    Let $\F \subseteq \E_1, \E_2 \subseteq \K$ be fields. Consider
    \[
        \bbL \vcentcolon= \left\{ \sum_{i=1}^n \, \alpha_i \beta_i \mid n \in \N, \alpha_i \in \E_1, \beta_i \in \E_2 \right\}.
    \]
    That is, $\bbL$ is the set of all finite sums of products of elements of $\E_1$ and $\E_2$.
    
    Suppose $d \vcentcolon= [\E_1 \colon \F] \cdot [\E_2 \colon \F] < \infty$. Then, $\bbL = \E_1\E_2$ and $[\bbL \colon \F] \leq d$. If $[\E_1 \colon \F]$ and $[\E_2 \colon \F]$ are coprime, then equality holds.
\end{prop}

\begin{proof}
    We leave it as an exercise to the reader to show that $\bbL$ is a subring of $\K$. Thus, $\bbL$ is an integral domain. Let $n \vcentcolon= [\E_1 \colon \F]$ and $m \vcentcolon= [\E_2 \colon \F]$. If $\{\alpha_1, \ldots, \alpha_n\}$ and $\{\beta_1, \ldots, \beta_m\}$ are $\F$-bases for $\E_1$ and $\E_2$, then the set $\left\{ \alpha_i \beta_j \mid 1 \leq i \leq n, 1 \leq j \leq m \right\}$ clearly spans $\bbL$ over $\F$. Hence $\dim_{\F}\bbL \leq mn = d$. In particular, $\dim_{\F}\bbL$ is finite, so that $\bbL$ is a field by \Cref{prop:finite-dimensional-vector-space-implies-ID-is-field}. 
    
    Lastly, note that $[\E_i \colon \F]$ divides $[\bbL \colon \F]$ by the \nameref{prop:tower}. In particular, if $\gcd(m,n) = 1$, then $mn \divides [\bbL \colon \F]$. Since $[\bbL \colon \F] \leq mn$, we are done.
    
    Diagrammatically, this is depicted as
\begin{center}
    

\begin{tikzcd}
                                                                     & \mathbb{K} \arrow[d, no head]                         &              \\
                                                                     & \mathbb{E}_1\mathbb{E}_2 \arrow[rd, "\leq n", no head] &              \\
\mathbb{E}_1 \arrow[ru, "\leq m", no head] \arrow[rd, "n"', no head] &                                                       & \mathbb{E}_2 \\
                                                                     & \mathbb{F} \arrow[ru, "m"', no head]                  &             
\end{tikzcd}
\end{center}
\end{proof}

\subsection{Splitting Fields}

\begin{defn}
    Let $\F$ be a field and $f(x) \in \F[x]$ be a non-constant polynomial of degree $n$ with leading coefficient $a \in \F^{\times}$. A field $\K \supseteq \F$ is called a \deff{splitting field of $f(x)$ over $\F$} if there exist (not necessarily distinct) $r_1, \ldots, r_n \in \K$ such that $f(x) = a(x-r_1)\cdots(x-r_n)$ and $\K = \F(r_1, \ldots, r_n)$. 
\end{defn}

\begin{ex}
    Consider $\F = \Q$, $f(x) = x^2 + 1 \in \Q[x]$, and $\K = \C$. Although $f(x)$ does factor linearly over $\C$ as $(x+\iota)(x-\iota)$, $\C$ is \textbf{not} a splitting field of $f(x)$ over $\Q$ since $\C \neq \Q(\iota, -\iota)$. On the other hand, $\C$ \emph{is} a splitting field of $f(x)$ over $\R$ since $\C = \R(\iota, -\iota)$.
\end{ex}

\begin{cor}
    Let $f(x) \in \F[x]$ be non-constant and let $\K$ be a splitting field of $f(x)$ over $\F$. Then, $\K/\F$ is an algebraic extension.
\end{cor}
\begin{proof}
    This follows trivially from \Cref{prop:field-generated-by-algebraic-is-finite-extension}.
\end{proof}

\begin{theorem} \label{thm:extension-field-that-contains-root}
    Let $\F$ be a field and let $f(x) \in \F[x]$ be non-constant. Then, there exists a field $\K \supseteq \F$ such that $f(x)$ has a root in $\K$.
\end{theorem}
\begin{proof}
    Let $g(x)$ be an irreducible factor of $f(x)$. Since $\F[x]$ is a principal ideal domain (\Cref{cor:F[x]-is-UFD-and-PID}), $g(x)$ is also a prime element (\Cref{prop:PID-prime-iff-irreducible}), so that $\langle g(x) \rangle$ is a prime ideal. Since $g(x)$ is non-zero, $\langle g(x) \rangle$ is also a maximal ideal (\Cref{prop:prime-is-maximal-PID}). Now, put $\K = \F[x]/\langle g(x) \rangle$. $\K$ is clearly a field by \Cref{prop:ideal-characterisation-using-quotient-ring}. $\K$ clearly contains $\F$ as a subfield via the identification $a \mapsto \overline{a}$, where the bar indicates the image in the quotient. Moreover, $\overline{x}$ is a root of $g(x)$ since $g(\overline{x}) = \overline{g(x)} = 0$ in the quotient.
\end{proof}

\begin{theorem}[Existence of Splitting Field] \label{thm:existence-of-splitting-field}
    Let $\F$ be a field. Any polynomial $f(x) \in \F[x]$ of positive degree has a splitting field. 
\end{theorem}
\begin{proof}
    Let $n \vcentcolon= \deg f(x)$. By \Cref{thm:extension-field-that-contains-root}, there exists a field $\F_1 \supseteq \F$ such that $f(x)$ has a root, say $a_1$, in $\F_1$. Now, 
    \[
        f(x) = (x-a_1) \cdot f_1(x)
    \]
    where $\deg f_1(x) = n-1$. Continuing inductively, we get fields
    \[
        \F_n \supseteq \cdots \supseteq \F_1 \supseteq \F
    \]
    with $a_i \in \F_i$ such that 
    \[
        f(x) = a(x-a_1)\cdots(x-a_n).
    \]
    Then, $\K = \F(a_1, \ldots, a_n) \subseteq \F_n$ is a splitting field.
\end{proof}