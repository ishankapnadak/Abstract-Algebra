\section{Group Theory}

\subsection{Introduction}

\begin{defn}
    Suppose $G$ is a nonempty set. A \deff{binary operation} on $G$ (or a \deff{law of composition}) is a function $\ast \colon G \times G \to G$. For any $a,b \in G$ we denote $\ast(a,b)$ as $a \ast b$ or simply $ab$.
\end{defn}

\begin{ex}
    Addition, subtraction and multiplication are binary operations on $\R$. Addition and subtraction are binary operations on $\R^{m \times n}$, the set of $m \times n$ real matrices. We denote the set of $n \times n$ real, invertible matrices by $GL_n(\R)$ (this is called the general linear group, as shall be discussed later). Multiplication is a binary operation on $GL_n(\R)$. Note however that addition is not a binary operation on $GL_n(\R)$ since the sum of two invertible matrices may be singular. $GL_1(\R)$ is denoted as $\R^{\times}$, the set of non-zero real numbers. Let $S$ be any non-empty set and let $\mathcal{M}$ denote the set of all functions from $S$ to $S$. Then, function composition is a binary operation on $\mathcal{M}$.
\end{ex}

\begin{defn}
    A binary operation $\ast$ on a set $G$ is said to be \deff{associative} if $(a \ast b) \ast c$ = $a \ast (b \ast c)$ for all $a,b,c \in G$.
\end{defn}

\begin{defn}
    A binary operation $\ast$ on a set $G$ is said to be \deff{commutative} if $a\ast b = b\ast a$ for all $a,b \in G$.
\end{defn}

\begin{prop}[Generalised Associative Law] \label{prop:gen-ass-law}
    Let $G$ be a set and let $\ast$ be an associative binary operation on $G$. For any $g_1, \ldots, g_n \in G$, the product $g_1 \ast \cdots \ast g_n$ is independent of how we bracket it. 
\end{prop}
\begin{proof}
    This is left as an exercise. The idea is to use induction on $n$. First show the basis. Then, assume that for any $k<n$, any bracketing of of $k$ elements $b_1 \ast \cdots \ast b_k$ can be reduced to $b_1 \ast (b_2 \ast ( \cdots b_k)) \cdots)$. Next, argue that $a_1 \ast \cdots \ast a_n$ can be reduced to $(a_1 \ast \cdots \ast a_k) \ast (a_{k+1} \ast \cdots \ast a_n)$ for some $k < n$. Apply the induction condition on each subproduct to complete the proof.
\end{proof}

\begin{defn}
    A \deff{group} is an ordered pair $(G,\ast)$ where $G$ is a set and $\ast$ is a binary operation on $G$ such that
    \begin{enumerate}
        \item $(a \ast b) \ast c = a \ast (b \ast c)$ for all $a,b,c \in G$, that is, $\ast$ is associative,
        \item there exists an element $e \in G$, called an \deff{identity} of $G$, such that $a \ast e = e \ast a = a$ for all $a \in G$, and
        \item for each $a \in G$, there is an element $a^{-1} \in G$, called an \deff{inverse} of $a$, such that $a \ast a^{-1}$ = $a^{-1} \ast a = e$.
    \end{enumerate}
\end{defn}

We say that $G$ is a group under $\ast$ if $(G,\ast)$ is a group. If $\ast$ is clear from context, we may simply say that $G$ is a group. We further say that $G$ is a \emph{finite group} if $G$ is a finite set. Note that any group is nonempty by virtue of the existence of an identity element.

\begin{defn}
    We say that a group $(G,\ast)$ is \deff{abelian} if $a \ast b = b \ast a$ for all $a,b \in G$.
\end{defn}


\begin{defn}
    Let $G$ be a group. We define the \deff{order} of $G$ as the cardinality of $G$, denoted by $\abs{G}$.
\end{defn}

\begin{ex}\phantom{hi}
\begin{enumerate}
\item $\Z, \C, \Q$ and $\R$ are all groups under the addition operation with $e = 0$ and $a^{-1} = -a$, for all $a$. $\Q^{\times}$, $\R^{\times}$, $\C^{\times}$, $\Q^+$, $\R^+$ are all groups under the multiplication operation with $e = 1$ and $a^{-1} = 1/a$, for all $a$. Note however that $\Z^{\ast}$ is not a group under the multiplication operation since the element $2$ (for instance) does not have an multiplicative inverse in $\Z^{\ast}$. We shall take the associative laws of these sets under addition and multiplication as given. 
    \item Rotation matrices in $2$-dimensions with multiplication also form a group. This is called the $SO_2(\R)$ group. This is the set of special orthogonal matrices - the set of $2 \times 2$ orthogonal matrices with determinant 1. On the other hand, the set of $2 \times 2$ orthogonal matrices forms another group, called the orthogonal group, $O_2(\R)$.
    \item Consider the group of non-zero complex numbers, $\C^{\times}$ under multiplication ($GL_1(\C)$). This also forms a group under multiplication. For any $n \in \N^+$, consider the $n^{\text{th}}$ root of unity, defined as
\[
    \omega_n = \cos \frac{2\pi}{n} + \iota \sin \frac{2\pi}{n}
\]
The set containing all powers of $\omega_n$ forms a finite group of order $n$ under complex multiplication, whose elements are precisely the $n$ roots of the polynomial $z^n = 1$. This group is called the \emph{cyclic group generated by $\omega_n$}, and is denoted as $\mu_n$.
\item For $n \in \N^+$, $\Z_n$ is an abelian group of order $n$ under the addition operation with $e = \overline{0}$ and the inverse of $\overline{a}$ defined as $\overline{-a}$. We denote this group as $\Z_n$. Notice that $\Z_n$ behaves similar to the cyclic group generated by $\omega_n$ and can be thought of as being generated by the equivalence class $\overline{1}$.

However, $\Z_n$ does not form a group under multiplication. This is because not all numbers have a multiplicative inverse modulo $n$. From number theory, we know that a number $a$ has a multiplicative inverse modulo $n$ if and only if $(a,n) = 1$. The set of equivalence classes $\overline{a}$ which have a multiplicative inverse modulo $n$ form an abelian group under multiplication. We denote this group as $\Z_n^{\times}$. The order of this group is equal to the number of integers between $1$ and $n$ which are co-prime with $n$. This is given precisely by Euler's totient function, $\varphi$. Thus, $\Z_n^{\times}$ forms an abelian group of order $\varphi(n)$ under multiplication. We sometimes also denote this group as $U_n$.
\end{enumerate}

\end{ex}


\medskip

\begin{theorem} \label{thm:group-basics}
    Let $G$ be a group under operation $\ast$. Then
    \begin{enumerate}
        \item The identity of $G$ is unique 
        \item For each $g \in G$, $g^{-1}$ is unique
        \item For each $g \in G$, $(g^{-1})^{-1} = g$
        \item For $a,b \in G$, $(a \ast b)^{-1} = b^{-1} \ast a^{-1}$
    \end{enumerate}
\end{theorem}

\begin{proof} \phantom{hi}
    \begin{enumerate} 
        \item Let $f$ and $g$ be two identities of $G$. We have $f \ast g = f$ and $f \ast g = g$. Thus $f=g$.
        \item Let $a,b \in G$ be two inverses of some $g \in G$ and let $e$ be the identity of $G$. We show that $b = a$.
        \begin{align*}
            b &= b \ast e \text{ (definition of $e$)} \\
            &= b \ast (g \ast a) \text{ (since $a$ is an inverse of $g$)} \\
            &= (b \ast g) \ast a \text{ (associativity)} \\
            &= e \ast a \text{ (since $b$ is an inverse of $g$)} \\
            &= a \text{ (definition of $e$)}
        \end{align*}
        \item We have $g^{-1} \ast g = g \ast g^{-1} = e$, implying that $(g^{-1})^{-1} = g$.
        \item Using the generalised associative law (\Cref{prop:gen-ass-law}) on $(a \ast b) \ast (b^{-1} \ast a^{-1})$ and $(b^{-1} \ast a^{-1}) \ast (a \ast b)$ gives the required result.
    \end{enumerate}
\end{proof}

\emph{Notation:} For any group ($G, \ast$), we denote $a \ast b$ as $ab$. For some group $G$, $g \in G$ and $n \in \Z^+$, we write $x \cdots x$ ($n$ times) as $x^n$. For $n<0, n \in \Z$, we define $x^n \vcentcolon= (x^{-1})^{-n}$, which is the same as $(x^{-n})^{-1}$ (Prove!) We usually denote the identity element of any group as $1$ and we define $x^0 \vcentcolon= 1$.

\begin{defn} Let $G$ be a group and let $x \in G$. Let $n$ be the smallest positive integer such that $x^n = 1$. $n$ is called the \deff{order} of $x$ and is denoted by $\abs{x}$. If no such positive power exists, we say that $x$ is of infinite order.
\end{defn}

\begin{prop} \label{thm:finite-group-finite-order}
    Any element of a finite group has finite order.
\end{prop}
\begin{proof}
    Let $G$ be a group and let $x \in G$. It suffices to show that $x^n = 1$ for some $n \in \N$. Note that $x^0, \ldots, x^{\abs{G}}$ are $\abs{G} + 1$ elements of $G$. Since the cardinality of $G$ is $\abs{G}$, we may conclude that two of these must be equal (pigeonhole principle). Thus, 
    \[
        x^n = x^m
    \]
    for some $0 \leq n < m \leq \abs{G}$. This gives us
    \[
        1 = x^{m-n}
    \]
    Since $m-n \in \N$, the claim follows.
\end{proof}

\subsection{Dihedral Groups}

We now look at importance class of groups whose elements are symmetries of geometric objects. The simplest objects to consider are regular $n$-gons. For each $n \in \Z^+$, $n \geq 3$, we let $D_{2n}$ be the set of symmetries of a regular $n$-gon. A symmetry is any rigid motion of the $n$-gon such that after this motion, the $n$-gon exactly covers the original $n$-gon. This can be thought of as first labelling $n$ vertices as $1, 2, \ldots, n$ and then describing a symmetry uniquely by the corresponding permutation $\sigma$ of $\{1,2,\ldots,n\}$. 

We make $D_{2n}$ a group by defining $st$ for $s,t \in D_{2n}$ to be the symmetry obtained by first applying $t$ then $s$. That is, if $s$ and $t$ have the permutations $\sigma$ and $\tau$ respectively on the vertices then $st$ has the permutation $\sigma \circ \tau$. 

We now show that $\abs{D_{2n}} = 2n$. Observe that vertex $1$ can be mapped to any one of the $n$ vertices. Let's say that it is mapped to vertex $i$. Now, since vertex $2$ is adjacent to vertex $1$, it must be mapped to either $i+1$ or $i-1$. The position of vertex $2$ fixes the entire permutation. Thus, we have $2n$ possible permutations, and so $\abs{D_{2n}} = 2n$. We call $D_{2n}$ the \emph{dihedral group of order $2n$}.

These $2n$ symmetries are $n$ rotations by $2\pi i/n$ radians about the center for $i=1,2, \ldots, n$ and the $n$ reflections about the $n$ lines of symmetry.

\medskip

Let $r$ be the clockwise rotation of the $n$-gon by $2\pi/n$ radians and let $s$ be the reflection symmetry that reflects the $n$-gon about the axis passing through vertex $1$ and the center. The following properties follow (proof is omitted):
\begin{enumerate}
    \item $1,r,r^2, \ldots, r^{n-1}$ are all distinct and $r^n = 1$. Thus, $\abs{r} = n$
    \item $\abs{s} = 2$
    \item $s \neq r^i$ for any $i$
    \item $sr^i \neq sr^j$ for all $0 \leq i,j \leq n-1$, $i \neq j$. Thus, we have
    \[
        D_{2n} = \left\{ 1, r, r^2,\ldots, r^{n-1}, s, sr, sr^2, \ldots sr^{n-1} \right\}.
    \]
    \item $rs = sr^{-1}$ (Since $r$ and $s$ do not commute, $D_{2n}$ is non-abelian) \footnote{Note that to claim $D_{2n}$ to be non-abelian, we need $r \neq r^{-1}$. This is true for $n \geq 3$.}
    \item $r^is = sr^{-i}$
\end{enumerate}

We conclude that all elements of $D_{2n}$ can be expressed uniquely as $s^kr^i$ where $k$ is $0$ or $1$ and $0 \leq i \leq n-1$. Moreover, identities (1), (2) and (6) will easily allow us to obtain this unique representation. Consider $n=12$. For example, we have
\[
    (sr^9)(sr^6) = s(r^9s)r^6 = s(sr^{-9})r^6 = s^2r^{-3} = r^{-3} = r^9
\]

Finally, another common way of writing the dihedral group, is as a presentation\footnote{A \emph{presentation} is another form of defining a group $G$. We have a set of generators, $S$, such that every element of $G$ can be written as a product of these generating elements. We also have a set of relations, $R$, among these generators. The group is then \emph{presented} as $\langle S \mid R \rangle$.} as follows
\[
    D_{2n} = \left\langle r,s \mid r^n = s^2 = 1 ; \, rs = sr^{-1} \right\rangle.
\]

\subsection{Quaternion and Heisenberg Groups}

The quaternion group is a group of order $8$, defined as follows
\[
    Q_8 = \left\{ \pm 1, \pm \I, \pm \J, \pm \K \right\}
\]
where each element is a $2 \times 2$ complex matrix of determinant $1$. Hence, this group lies within $GL_2(\C)$. The matrices are defined as follows
\[
    1 = \begin{bmatrix}
        1 & 0 \\
        0 & 1
    \end{bmatrix} \quad
    \I = \begin{bmatrix}
        \iota & 0 \\
        0 & -\iota
    \end{bmatrix} \quad
    \J = \begin{bmatrix}
        0 & 1 \\
        -1 & 0
    \end{bmatrix} \quad
    \Kh = \begin{bmatrix}
        0 & \iota \\
        \iota & 0
    \end{bmatrix}
\]
In $Q_8$, we have the following relations
\[
    {\I}^2 = {\J}^2 = {\Kh}^2 = -1
\]
\[
    \I\J = \Kh = -\J\I
\]
\[
    \J\Kh = \I = -\Kh\J
\]
\[
    \Kh\I = \J = -\I\Kh
\]
\[
    \I\J\Kh = -1
\]

\medskip

The Heisenberg group is a group of $3\times 3$ upper-triangular matrices, defined as follows
\[
    H(\R) = \left\{ \begin{bmatrix}
        1 & a & b \\
        0 & 1 & c \\
        0 & 0 & 1
    \end{bmatrix} \mid a,b,c \in \R \right\}.
\]

$H(\R)$ forms an infinite non-abelian  group under matrix multiplication, where each element has determinant $1$.

\subsection{Symmetric Groups}

Let $\Omega$ be any non-empty set and $S_{\Omega}$ be the set of all bijections from $\Omega$ to $\Omega$ (or permutations of $\Omega$). The set $S_{\Omega}$ is a group under function composition, $\circ$, since function composition is associative, the identity is the identity map on $\Omega$ and every bijection has an inverse bijection. In the case where $\Omega = \{1,\ldots,n\}$, we denote $S_{\Omega}$ as $S_n$. $S_n$ is called the \emph{symmetric group of order n}. It is easy to show that $\abs{S_n} = n!$. We now illustrate a convenient notation to write elements of $S_n$, called the \emph{cycle decomposition}. A \emph{cycle} is a string of integers that cyclically permutes the integers of this string, leaving all other integers fixed. For example, $(a_1 \, a_2 \, \ldots \, a_k)$ sends $a_1$ to $a_2$, $a_2$ to $a_3, \ldots, a_{k-1}$ to $a_k$ and $a_k$ to $a_1$. In general, any element $\sigma$ of $S_n$ can be rearranged and written as $k$ (disjoint) cycles as 
\[
    \sigma = (a_1 \, \ldots \, a_{m_1}) (a_{m_1+1} \, \ldots \, a_{m_2}) \ldots (a_{m_{k-1}+1} \, \ldots \, a_{m_k})
\]


To find where an element $i$ is sent to by a permutation, we simply need to find the element written after $i$ in the cycle decomposition. Any permutation $\sigma$ can be easily written as its cycle decomposition using the following algorithm. 
\begin{enumerate}
    \item To start a new cycle, pick the smallest number in $\{1, \ldots, n\}$ that has not appeared in a previous cycle. Call it $a$. Begin the new cycle $(a$
    \item Let $\sigma(a) = b$. If $b=a$, close the cycle and return to step $1$. If $b \neq a$, write $b$ next to $a$ so that the cycle becomes $(a \, b$
    \item Let $\sigma(b) = c$. If $c = a$, close the cycle and return to step 1. If $c \neq a$, write $c$ next to $b$ and repeat this step with $c$ as $b$ until the cycle closes.
\end{enumerate}

The \emph{length} of a cycle is the number of integers appearing in the cycle. A cycle of length $l$ is called an $l$-cycle. (Notice that an $l$-cycle has order $l$) By convention, we omit $1$-cycles. Thus, if some element $i$ does not in a cycle decomposition of a permutation, it is understood that the permutation fixes $i$. The identity permutation is written as $1$. The final step in the algorithm is thus to remove all $1$-cycles. Note that
\[
    (1 \, 3) \, (1 \, 2) = (1 \, 2 \, 3) \text{ and } (1 \, 2) \, (1\, 3) = (1 \, 3 \, 2)
\]  
This shows that $S_n$ is \emph{non-abelian} for all $n \geq 3$. Note that since disjoint cycles permute elements in disjoint sets, disjoint cycles commute. Thus, rearranging the cycles in any product of disjoint cycles does not change the permutation. 


\begin{rem}
    We define an equivalence relation on $\Omega$ (any general non-empty set), with $a \sim b$ if $b = \sigma^k(a)$ for some $k$. Here $\sigma^k$ denotes the permutation $\sigma$ composed $k$ times. It is easy to verify that this is an equivalence relation. Each disjoint cycle in the cycle decomposition of $\sigma$ represents an equivalence class of $\sim$. (Verify!)
\end{rem}

Note that every symmetry transformation of an equilateral triangle can be associated with a unique permutation of the vertices. Likewise, every permutation of the vertices of an equilateral triangle can be associated with (the same) symmetry transformation. We see that $D_6$ and $S_3$ are essentially the same group. This will be made more precise when we discuss isomorphisms.

\begin{comment}
\begin{theorem}
    Every $\sigma \in S_{\Omega}$ can be expressed as a product of disjoint cycles.
\end{theorem}
\begin{proof}
    We define a relation on $S_{\Omega}$ with $a \sim b \iff b = \sigma^k(a)$ for some $k$. One can verify that this defines an equivalence relation on $S_{\Omega}$. Thus, $S_{\Omega}$ is partitioned into equivalence classes $C_{i_1}, \ldots, C_{i_t}$ where $C_{i_k}$ is the equivalence class of $i_k$. In fact, $C_{i_k}$ corresponds to the cycle $i_k, \sigma(i_k), \sigma^2(i_k), \ldots$. This list cannot be infinite since $S_{\Omega}$ is a finite set. Similarly, each equivalence class forms its own cycle. Since the equivalence classes are disjoint, their resulting cycles are also disjoint. Let these $t$ cycles be $\tau_1, \ldots, \tau_t$. We wish to show that $\tau_1 \cdots \tau_t = \sigma$.
\end{proof}

\begin{prop}
        If $\tau_1$ and $\tau_2$ are two disjoint cycles, then $\tau_1 \tau_2 = \tau_2 \tau_1$.
    \end{prop}
    \begin{proof}
        Suppose $\omega \in \Omega$ such that $\tau_1(\omega) = \tau_2(\omega) = \omega$. For this case, it is easy to verify that $\tau_1\tau_2(\omega) = \tau_2\tau_1(\omega) = \omega$. For any $\omega \in \Omega$, if we have $\tau_1(\omega) \neq \omega$ then we must have $\tau_2(\omega) = \omega$ since they are disjoint. Now, $\tau_1\tau_2(\omega) = \tau_1(\omega)$. Also, $\tau_1(\omega)$ does not occur in the cycle of $\tau_2$. Thus, $\tau_2\tau_1(\omega) = \tau_1(\omega)$. A similar argument can be used when $\tau_2(\omega) \neq \omega$. Hence, the claim follows.
    \end{proof}
\end{comment}

\subsection{Conjugacy}

Recall from linear algebra that an $n \times n$ real symmetric matrix, $A$ can be diagonalised. That is, if $\lambda_1, \ldots, \lambda_n$ are the eigenvalues of $A$, then $A \sim \Lambda$, where $\Lambda$ is a diagonal matrix containing the eigenvalues. That is, $C A C^{-1} = \Lambda$ for an \emph{orthogonal} matrix, $C$, whose column vectors are the corresponding eigenvectors of $A$. This very idea can be made more abstract and applied to groups, in general. 

\begin{defn}
    Let $G$ be a group and $g,h \in G$. We say that $g$ is a \deff{conjugate} of $h$ if there exists an $x \in G$ such that $h = xgx^{-1}$.
\end{defn}

\begin{prop} \label{prop:conjugacy-equivalence}
    Conjugacy is an equivalence relation on the group $G$
\end{prop}
\begin{proof} We prove reflexivity, symmetry and transitivity.
    \begin{enumerate}
        \item For reflexivity, we may take $x$ to be identity to give us that $h \sim h$ for all $h \in G$.
        \item Suppose $h \sim g$. Then, $h = xgx^{-1}$ for some $x \in G$. Left-multiplying by $x^{-1}$ and right-multiplying by $x$, we get $g = x^{-1}hx = x^{-1}h(x^{-1})^{-1}$. Since, $x^{-1} \in G$, we have that $g \sim h$, proving symmetry.
        \item Suppose $h \sim g$ and $l \sim h$. Then, there exist $x,y \in G$ such that $h = xgx^{-1}$ and $l = yhy^{-1}$. Substituting $h$, we see that $l = (yx) g (x^{-1} y^{-1}) = (yx) g (yx)^{-1}$. Since $yx \in G$, we have that $l \sim g$, proving transitivity.
    \end{enumerate}
\end{proof}

The equivalence classes of the conjugacy relation are called \emph{conjugacy classes}. Thus, all of $G$ is a disjoint union of conjugacy classes. We see that for any group $G$, the identity element is the only element in its conjugacy class. If $G$ is abelian, then $gag^{-1} = a$ for all $a, g \in G$. Thus, each element in an abelian group is part of a unique conjugacy class. We shall typically denote the conjugacy class of $g \in G$ as $C(g)$. 

\begin{prop} \label{prop:same-conjugacy-same-order}
    Let $G$ be a group and let $g,h \in G$ belong to the same conjugacy class, i.e, $g \sim h$. Then, $\abs{g} = \abs{h}$
\end{prop}
\begin{proof}
    Since $g \sim h$, we know that $g = xhx^{-1}$ for some $x \in G$. Now
    \[
        g^n = (xhx^{-1})^n = xh^nx^{-1}
    \]
    Thus, $g^n = 1 \iff h^n = 1$.
\end{proof}

We want to understand what are the conjugacy classes in $S_n$. We will first begin by analysing the conjugacy classes of $S_3$. Using our familiar cycle decomposition notation, we may define $S_3$ as follows
\[
    S_3 = \left\{ 1, (1 \, 2), (1 \, 3), (2 \, 3), (1 \, 2 \, 3), (1 \, 3 \, 2) \right\}.
\]

The conjugacy class of a $2$-cycle cannot contain any $3$-cycle or the identity element (since they have different orders). With a little bit of work, we can show that the number of conjugacy classes are precisely $3$ - the identity, the $2$-cycles and the $3$-cycles. Let us relate these numbers with partitions of natural numbers. We first define what are partitions.

\begin{defn}
    Let $n \in \N^+$. A \deff{partition} of $n$ is a tuple $\lambda = (\lambda_1, \ldots, \lambda_l)$ of positive integers $\lambda_1 \geq \ldots \geq \lambda_l$ such that $\lambda_1 + \cdots + \lambda_l = n$.
\end{defn}


The number of partitions of $3$ is equal to $3$ and these are given by $(1,1,1)$, $(2,1)$ and $(3)$. This number is exactly equal to the number of conjugacy classes of $S_3$. For a general symmetric group, $S_n$, the number of conjugacy classes is given precisely by the number of partitions of $n$. 

\medskip

Suppose $\tau = (i_1, \ldots, i_k)$ is a $k$-cycle. Suppose $\sigma \in S_n$. We want to show that 
\[
    \sigma\tau\sigma^{-1} = \left( \sigma(i_1), \ldots, \sigma(i_k) \right).
\]
We have
\begin{align*}
        \sigma\tau\sigma^{-1} (\sigma(i_1)) &= \sigma\tau(\sigma^{-1}\sigma)(i_1) \\
        &= \sigma\tau(i_1) \\
        &= \sigma(i_2)
\end{align*}
Similarly, we show that the two permutations have the same effect on all $k$  elements, $\sigma(i_1), \ldots, \sigma(i_k)$. Now, suppose $x \notin \{\sigma(i_1), \ldots , \sigma(i_k)\} \iff \sigma^{-1}(x) \neq i_s$ for any $s = 1, \ldots, k$. We have
\begin{align*}
    \sigma\tau\sigma^{-1}(x) &= \sigma\sigma^{-1}(x) \text{ (since $\sigma^{-1}(x)$ is a fixed point of $\tau$)} \\
    &= x
\end{align*}

Hence, the conjugate of a $k$-cycle is also a $k$-cycle. We may now explicitly construct a permutation $\sigma$ to show that both $3$-cycles are conjugates. Similarly, we may show that all $2$-cycles form a conjugacy class.

\medskip

Now, we look at conjugates of any permutation in $S_n$. Suppose $\sigma \in S_n$ and $\sigma = \tau_1\cdots\tau_k$ where $\tau_1,\ldots,\tau_k$ are disjoint cycles of length at least $2$. Let $\gamma \in S_n$. We have
\begin{align*}
    \gamma\sigma\gamma^{-1} &= \gamma(\tau_1 \cdots \tau_k)\gamma^{-1} \\
    &= (\gamma\tau_1\gamma^{-1}) \cdots (\gamma\tau_k\gamma^{-1})
\end{align*}
Here $\gamma\tau_i\gamma^{-1}$ is a conjugate of $\tau_i$. Hence, the cycle structure of the conjugate of the permutation remains the same as the original permutation. Moreover, the conjugates of disjoint cycles are also disjoint. 

For example, consider $S_5$ and the permutation
\[
    \tau = (1) (2) (3 \, 4 \, 5)
\]
This corresponds to the partition $(3,1,1)$ of $5$. Moreover, given any $\sigma \in S_5$, the conjugate of $\tau$ will be
\[
    (\sigma(1)) (\sigma(2)) \left( \sigma(3) \, \sigma(4) \, \sigma(5) \right)
\]
Thus, the conjugacy class of $\tau$ is precisely the set
\[
    \left\{ (\sigma(1)) (\sigma(2)) \left( \sigma(3) \, \sigma(4) \, \sigma(5) \right) \mid \sigma \in S_5 \right\}
\]
Thus, corresponding to every partition of $5$, we have a unique conjugacy class of $S_5$. In general, corresponding to every partition of $n$, we have a unique conjugacy class of $S_n$. Thus, the number of conjugacy classes in $S_n$ is $p(n)$, the number of partitions of $n$.

\subsection{Odd and Even Permutations}

\begin{prop} \label{prop:perm-prod-of-transpositions}
    Every permutation can be written as a product of transpositions ($2$-cycles)
\end{prop}
\begin{proof}
    We first express a $k$-cycle as a product of $2$-cycles. Consider the $k$-cycle $(a_1 \, \ldots \, a_k)$. Verify that we have
    \[
        (a_1 \, \ldots \, a_k) = (a_1 \, a_k) (a_1 \, a_{k-1}) \ldots (a_1 \, a_3) \, (a_1 \, a_2)
    \]
    We have thus shown that every $k$-cycle is a product of transpositions. We also know that any permutation can be written as a product of disjoint cycles. Hence, the claim follows.
\end{proof}

\begin{defn}
    A permutation $\sigma \in S_n$ is called \deff{even (odd)} if $\sigma$ is a product of an even (odd) number of transpositions
\end{defn}

We must prove that this is indeed a well-defined notion (that is, every permutation must either be an even permutation or an odd permutation). Consider the \emph{Vandermonde} polynomial, defined as
\[
    P(x_1, \ldots, x_n) = \prod_{1 \leq j < i \leq n} (x_i - x_j)
\]
Consider a permutation $\sigma \in S_n$. We define
\[
    \sigma P \vcentcolon= \prod_{1 \leq j < i \leq n} (x_{\sigma(i)} - x_{\sigma(j)})
\]
If $\sigma$ is a transposition, then $\sigma P = -P$. Thus, if $\sigma$ is an even permutation, we see that $\sigma P = P$ whereas if $\sigma$ is an odd permutation, we have $\sigma P = -P$. Thus, $\sigma$ must either be even or odd, as determined by its effect on $P$. Using this idea, we also trivially see that the product of two even permutations is even, the product of two odd permutations is even and the product of an even permutation and an odd permutation is odd. Moreover, the inverse of an even permutation is also an even permutation. The identity permutation is also an even permutation. We thus see that the set of even permutations forms a group by itself! We call this group $A_n$, the \emph{alternating group of degree $n$}. 

\begin{prop} \label{prop:order-of-An}
    The order of $A_n$ is $n!/2$.\footnotemark
\end{prop} 
\footnotetext{This assumes $n > 1$. For $n=1$, the group $S_n$ is the trivial group, and so is $A_n$. For this case, we have $\abs{A_n} = \abs{S_n} = 1$.}
\begin{proof}
    To show this, we set up a bijection between the set of even permutations and the set of odd permutations. Let $Z = S_n \setminus A_n$. We define a map $\varphi \colon A_n \to Z$, defined by $\varphi(\sigma) = (1 \, 2) \, \sigma$. This is a one-to-one map from $A_n$ to $Z$. This is also an onto map since given any $\tau \in Z$, we have $\varphi\left((1 \, 2) \, \tau\right) = \tau$. Thus, $\varphi$ is a bijection and hence $\abs{A_n} = \abs{Z}$. But we know that $\abs{S_n} = n! = \abs{A_n} + \abs{Z}$ (since $A_n$ and $Z$ are disjoint). This gives us that $\abs{A_n} = n!/2$.
\end{proof}

\begin{rem}
    The group $A_5$ is fundamental in proving that there exists a quintic polynomial which is not solvable by radicals.
\end{rem}

\subsection{Subgroups and Cyclic Groups}


\begin{defn}
    Let $G$ be a group. A subset $H$ of $G$ is a \deff{subgroup} of $G$ if $H$ is non-empty and closed under products and inverses. That is, $x,y \in H$ implies that $x^{-1} \in H$ and $xy \in H$. If $H$ is a subgroup of $G$, we write $H \leq G$.
\end{defn}

\begin{ex}
    \phantom{hi}
    \begin{enumerate}
        \item $A_3$ is a subgroup of $S_3$.
        \item $SL_n(\R)$, the group of $n \times n$ real matrices with determinant $1$, is a subgroup of $GL_n(\R)$ (under matrix multiplication).
        \item $O_n(\R)$ is a subgroup of $GL_n(\R)$ while $SO_n(\R)$ is a subgroup of $O_n(\R)$.
        \item The set of complex numbers of unity magnitude, denoted as $S^1$, forms a group under multiplication and is in fact a subgroup of $\C^{\times}$. The cyclic group generated by $\omega_n$, which we denote as $\mu_n$, is a (finite) subgroup of $S^1$.
    \end{enumerate}
\end{ex}


\begin{theorem}[Subgroup Criterion] \label{thm:subgroup-criterion}
    A subset $H$ of a group $G$ is a subgroup of $G$ if and only if
    \begin{enumerate}
        \item $H \neq \emptyset$.
        \item for all $x,y \in H$, $xy^{-1} \in H$.
    \end{enumerate}
\end{theorem}
\begin{proof}
    We only prove the converse. Let $x$ be any element of $H$ (such an element exists since $H \neq \emptyset$). We have $xx^{-1} \in H \implies 1 \in H$. For any $h \in H$, we have $1h^{-1} \in H \implies h^{-1} \in H$. Thus, $H$ is closed under inverses. For any $x,y \in H$, we know that $y^{-1} \in H$, and thus, $x(y^{-1})^{-1} \in H \implies xy \in H$. Hence, $H$ is also closed under multiplication.
\end{proof}

Let $G$ be any group and $g \in G$. We define the subgroup \emph{generated} by $g$ to be the smallest subgroup of $G$ containing $g$. We leave it as an exercise to verify that this is the group $\langle g \rangle \vcentcolon= \{ 1, g^{\pm 1}, g^{\pm 2}, \ldots \}$. Groups generated by a single element are called \emph{cyclic groups}.

\begin{prop} \label{prop:cyclic-structure}
    Suppose $H$ is a cyclic group generated by $x$, $H = \langle x \rangle$. If the order of $H$ is infinite, then $H = \left\{ 1, x^{\pm 1}, x^{\pm 2}, \ldots \right\}$, all of which are distinct elements. If $H$ is of order $n$, then the order of $x$ is also $n$ and $H = \{ 1, x, \ldots , x^{n-1} \}$.
\end{prop}
\begin{proof}
    Suppose the order of $H$ is infinite and $H$ is generated by $x$. All we need to show is that every power of $x$ is distinct. Suppose that $x^m = x^n$ for some $m > n$. This gives us $x^{m-n} = 1$. Let $d = m-n$, giving us $x^d = 1$. If $l \in \Z$, then by the \nameref{prop:div_algo}, $l = dq + r$ where $q \in \Z$ and $0 \leq r < d$. Now
    \[
        x^l = x^{dq+r} = (x^d)^q x^r = x^r
    \]
    Hence, every integral power of $x$ is $x^r$ for some $0 \leq r < d$. Thus, $H$ is finite, which is a contradiction. Hence, $x^m \neq x^n$ for $m \neq n$.
    
    \medskip
    
    Suppose $H = \langle x \rangle$ and $\abs{H} = n$. Since $H$ is finite, $\{1, x^{\pm 1}, x^{\pm 2}, \ldots\}$ is a finite list. As proved before, $x^d = 1$ for some $d \in \N^+$. Let $m$ be the smallest positive integer such that $x^m = 1$ (such a number exists because of \nameref{rem:wop}). Thus $\abs{x} = m$. This means that $\{1, x,\ldots, x^{m-1}\}$ is a group, which is precisely the same as $H$. Equating the number of elements, we get $m = n$. Thus, $\abs{x} = n$.
\end{proof}

\begin{prop} \label{prop:order_xa}
    Let $H = \langle x \rangle$. 
    \begin{enumerate}
        \item If $\abs{x}$ is infinite, then $\abs{x^a}$ is also infinite for any $a \neq 0$.
        \item If $\abs{x} = n$, then 
        \[
            \abs{x^a} = \frac{n}{\gcd(a,n)}
        \]
    \end{enumerate}
\end{prop}
\begin{proof} 
    The first part is rather trivial to prove and is left as an exercise. We now prove the second part. Let $d = (a,n)$. We have
    \[
        \left( x^a \right)^{n/d} = \left( x^n \right)^{a/d}
    \]
    Since the order of $x$ is $n$ and $a/d$ is an integer, we see that $\left( x^a \right)^{n/d} = 1$. It is also not too difficult to show that no $m < n/d$ satisfies $(x^a)^m = 1$.
\end{proof}

 Recall \Cref{cor:additive_subgroup_nZ}, which states that any subset of $\Z$ that is closed under inverses and addition must be of the form $n\Z$ for some $n \in \Z$. In other words, any subgroup of $\Z$ is of the form $n\Z$. Notice that $\Z$ is a cyclic group, generated by $1$ or $-1$, and all its subgroups are also cyclic subgroups. This idea extends to all cyclic groups, as stated next.

\begin{prop} \label{prop:cylic-subgroup-cyclic}
    Suppose $H$ is a cyclic group and $K \leq H$. Then, $K$ is cyclic.
\end{prop}
\begin{proof}
    If $K$ is the trivial subgroup (containing only the identity), it is clearly cyclic. Suppose $K$ is non-trivial. Then, there is a non-zero integer $n$ for which $x^n \in K$. Since $K$ is closed under inverses, $x^{-n} \in K$. Hence, there exists a $d \geq 1$ such that $x^d \in K$. Let $d$ be the smallest such integer (such a $d$ exists because of \nameref{rem:wop}). We claim that $K = \langle x^d \rangle$. This is easily proven using the division algorithm, and is left as an exercise.
\end{proof}

\begin{theorem} \label{thm:cyclic-subgroup-divisor}
    Let $H = \langle x \rangle$ be a finite cyclic subgroup of order $n$ and $m \in \N^+$. Then, $H$ has a subgroup of order $m$ if and only if $m \divides n$. Moreover, for each divisor $m$ of $n$, there is exactly one subgroup of order $m$ in $H$. (Alternatively, there is a one-one correspondence between subgroups of $H$ and divisors of $n$).
\end{theorem}
\begin{proof}
    We know that $\abs{x} = n$ and that every subgroup of $H$ is cyclic. Suppose $K = \langle x^a \rangle$ with $a \geq 1$. \Cref{prop:order_xa} tells us that $\abs{x^a} = n/d$ where $d = \gcd(a,n)$. Thus, the subgroup $\langle x^a \rangle$ is mapped to the divisor $\gcd(a,n)$. Conversely, assume that $d$ is a divisor of $n$. Then, 
    \[
        \abs{x^{n/d}} = \frac{n}{\gcd\left( \frac{n}{d}, n \right)} = \frac{n}{n/d} = d
    \]
    Hence, we map the divisor $d$ to the subgroup generated by $x^{n/d}$. It remains to show that for each divisor $d$ of $n$, there exists a unique subgroup of order $d$. 

    Suppose $K \leq H$ and $\abs{K} = d$. We must show that $K = \langle x^{n/d} \rangle$. Since $K$ is cyclic, there is a $b \in \N^+$ such that $K = \langle x^b \rangle$. This implies that $\abs{K} = n/\gcd(b,n)$. Thus, $\gcd(b,n) = n/d \implies n/d \divides b$. Thus $b = (n/d) \cdot c$ for some $c \in \Z$. Now,
    \[
        x^b = x^{c(n/d)} \in \langle x^{n/d} \rangle
    \]
    Thus, $K \leq \langle x^{n/d} \rangle$. However, order of both these groups are equal to $d$, which concludes that that the two groups are equal, or $K = \langle x^{n/d} \rangle$.
\end{proof}

Using this theorem, we can in fact find the total number of subgroups of $H$ (total number of divisors of $n$) and also construct each of these subgroups with the help of the prime-factorisation of $n$.

\begin{prop} \label{prop:cyclic-generators}
    Suppose $H = \langle x \rangle$. 
    \begin{enumerate}
        \item If $H$ is infinite, then $H$ has only one other generator, $x^{-1}$.
        \item If $\abs{H} = n$ then $x^a$ generates $H$ if and only if $\gcd(a,n) = 1$.
    \end{enumerate}
\end{prop}
\begin{proof}
First consider that $H$ is infinite. We have shown that $H = \{1, x^{\pm 1}, x^{\pm 2}, \ldots \}$, all of which are distinct elements. Suppose $x^a$ also generates $H$. Then, $H = \{ 1, x^{\pm a}, x^{\pm 2a}, \ldots \}$. Comparing the two forms, $x = x^{na}$ for some $n \in \Z^{\ast}$. This gives us $x^{na - 1} = 1$. If $na - 1$ is non-zero then $x$ generates a finite cyclic group, which is a contradiction. Hence, $na = 1$. This gives us precisely the two solutions $a = \pm 1$. Thus, any infinite cyclic group has only two possible generators $x$ or $x^{-1}$. We leave it to the reader to verify that $x^{-1}$ indeed generates $H$.

\medskip

Now suppose $H$ is finite. $\abs{H} = n \implies \abs{x} = n$. We know from \Cref{prop:order_xa} that $\abs{x^a} = n/d$ where $d = \gcd(a,n)$. If $x^a$ also generates $H$ then $\abs{x^a} = n$. This gives us $d = 1$. Conversely, suppose that $\gcd(a,n) = 1$. Then, $\abs{x^a} = n$ and hence, $x^a$ generates a subgroup of H of order $n$. However, the only subgroup of $H$ of order $n$ is $H$ itself. Hence, $x^a$ generates $H$. This also gives us that the number of generators of $H$ are $\varphi(n)$.
\end{proof}

\begin{ex} \phantom{hi}
\begin{enumerate}
    \item Consider the multiplicative group $\Z_2^{\times} = \left\{ \overline{1}, \overline{5}, \overline{7}, \overline{11} \right\}$. This is a group of order $4$. However, all its elements have order either $1$ or $2$. Hence, this group is not cyclic since a cyclic group of order $4$ must have an element of order $4$.
    
    \item Consider the real Heisenberg group, $H(\R)$ which is a subgroup of $SL_3(\R)$. Consider a matrix
\[
    M = \begin{bmatrix}
    1 & a & b \\
    0 & 1 & c \\
    0 & 0 & 1
    \end{bmatrix}
\]
where $M$ is not an identity matrix. We may decompose $M$ as
\[
    M = \underbrace{\begin{bmatrix}
    1 & 0 & 0 \\
    0 & 1 & 0 \\
    0 & 0 & 1
    \end{bmatrix}}_{I} + \underbrace{\begin{bmatrix}
    0 & a & b \\
    0 & 0 & c \\
    0 & 0 & 0
    \end{bmatrix}}_{N}
\]  
One may verify that $N^3 = 0$ and
\[
    N^2 = \begin{bmatrix}
    0 & 0 & ac \\
    0 & 0 & 0 \\
    0 & 0 & 0
    \end{bmatrix}
\]
Now, 
\[
    M^n = (I + N)^n = I + nN + \binom{n}{2} N^2 
\]
For this matrix to generate a finite cyclic group, we need $M^n = I$ for some $n$. This gives us $a = b = c = 0$, which is a contradiction since $M \neq I$. Thus, every non-identity matrix in $H(\R)$ generates an infinite cyclic group.
\end{enumerate}
\end{ex}

