\section{Polynomial Rings}

\subsection{Definitions}

For this section, whenever we talk about rings, we assume commutative rings. Recall that the polynomial ring $R[x]$ in the indeterminate $x$ with coefficients in the ring $R$ is defined as the set of all formal sums $a_nx^n + \ldots a_0$ with $n \geq 0$ and each $a_i \in R$. If $a_n \neq 0$, then $a_n x^n$ is the leading term, $a_n$ is the leading coefficient, and the degree of the polynomial is $n$. We define the leading coefficient of the zero polynomial as zero. If $a_n = 1$, we call the polynomial monic. With addition and multiplication of polynomials defined the usual way, $R[x]$ is a commutative ring that borrows its identity from $R$ itself. Moreover, we identify $R$ with the subring of constant polynomials. We now state, without proof, a proposition that summarises a bunch of results from \Cref{sec:poly}.

\begin{prop} \label{prop:poly-summary}
    Let $R$ be an integral domain. Then,
    \begin{enumerate}
        \item $\deg (p(x) q(x)) = \deg p(x) + \deg q(x)$ for all $p(x), q(x) \in R[x]$,
        \item the units of $R[x]$ are just the units of $R$, and
        \item $R[x]$ is an integral domain.
    \end{enumerate}
\end{prop}

Recall also that if $R$ is an integral domain, then we denote by $R(x)$, the field of fractions of $R[x]$ (the field of rational functions in $x$ with coefficients in $R$), which consists of all quotients of the form $\frac{p(x)}{q(x)}$ where $p(x), q(x) \in R[x]$ and $q(x)$ is non-zero.

\begin{prop} \label{prop:prime-ideals-maintained}
    Let $I$ be an ideal of the ring $R$ and let $\langle I \rangle = I[x]$ denote the ideal of $R[x]$ generated by $I$ (the set of polynomials with coefficients in $I$). Then, 
    \[
        R[x] / \langle I \rangle \cong (R/I)[x].
    \]
    In particular, if $I$ is a prime ideal of $R$, then $\langle I \rangle$ is a prime ideal of $R[x]$.
\end{prop}
\begin{proof}
    We have a natural map $\varphi \colon R[x] \to (R/I)[x]$ obtained by reducing each coefficient of a polynomial in $R[x]$ modulo $I$. Moreover, $\varphi$ is a ring homomorphism. Observe that $\ker\varphi = I[x] = \langle I \rangle$ and thus, $R[x]/\langle I \rangle \cong (R/I)[x]$ by \Cref{thm:ring-iso-1}, proving the first part. If $I$ is a prime ideal of $R$, then $R/I$ is an integral domain by \Cref{prop:ideal-characterisation-using-quotient-ring}, and hence, by \Cref{prop:poly-summary}, $(R/I)[x]$ is an integral domain. Once again, by \Cref{prop:ideal-characterisation-using-quotient-ring}, we conclude that $\langle I \rangle$ is a prime ideal of $R[x]$.
\end{proof}

\begin{defn}
    The \deff{polynomial ring in variables $x_1, \ldots , x_n$} with coefficients in $R$, denoted as $R[x_1, \ldots, x_n]$ is defined inductively as 
    \[
        R[x_1, \ldots, x_n] \vcentcolon= R[x_1, \ldots, x_{n-1}][x_n].
    \]
\end{defn}

A more concrete formulation of this idea is as follows. 

\begin{defn}
   A polynomial in $n$ variables $x_1, \ldots, x_n$ with coefficients in a commutative ring $R$ is an expression of the form
   \[
        \sum \, a_{d_1, \ldots, d_n} \, x_1^{d_1} \cdots x_n^{d_n}
   \]   
   where the summation is over a finite set of $n$-tuples $(d_1, \ldots, d_n)$ in $\N^n$ where $a_{d_1, \ldots, d_n} \in R$ for every such $n$-tuple.
\end{defn}

\begin{defn}
   Let $R$ be a commutative ring and let $\Lambda$ be a finite subset of $\N^n$. Let $f(x_1, \ldots, x_n)$ be a polynomial in $x_1, \ldots, x_n$ of the form
   \[
        f(x_1, \ldots, x_n) = \sum_{(d_1,\ldots,d_n)\,\in\, \Lambda} \, a_{d_1, \ldots, d_n} \, x_1^{d_1} \cdots x_n^{d_n}.
   \]
   Then, 
   \[
        a_{d_1,\ldots,d_n} \, x_1^{d_1} \cdots x_n^{d_n}
   \]
   is called a \deff{term} of the polynomial $f(x_1, \ldots,x_n)$ provided $a_{d_1,\ldots,d_n} \neq 0$. Moreover, $d_i$ is called the \deff{degree} of $x^i$ in the above term, and $d \vcentcolon= d_1+\ldots+d_n$ is called the \deff{degree} of this term. We call the $n$-tuple $(d_1, \ldots, d_n)$ the \deff{multidegree} of the term.
\end{defn}

For brevity, we represent a polynomial in $n$ variables, $f(x_1, \ldots, x_n)$ as simply $f$.

Two polynomials are equal if and only if they have the same terms. The zero polynomial is defined as the polynomial having no terms. Since any non-zero polynomial must have at least one term, we can define the degree of such polynomials as follows.
\begin{defn}
   Let $f$ be a non-zero polynomial in $x_1,\ldots,x_n$. The \deff{degree} or \deff{total degree} of $f$ is defined as
   \[
        \deg f \vcentcolon= \max\left\{ d_1 + \ldots + d_n \mid (d_1, \ldots, d_n) \in \Lambda \text{ and } a_{d_1, \ldots, d_n} \neq 0 \right\}.
   \]
   As in the case of single variable polynomials, we define the degree of the zero polynomial as $-\infty$.
\end{defn}

\begin{defn}
   Let $f$ be a non-zero polynomial in $x_1, \ldots, x_n$. If every term of the polynomial has the same degree $d$, then $f$ is said to be a \deff{homogeneous} polynomial of degree $d$.
\end{defn}

\begin{defn}
   A polynomial that has a single term with coefficient $1$, of the form $x_1^{i_1} \cdots x_n^{i_n}$ is called a \deff{monomial}.
\end{defn}



With this, we may think of a polynomial as a finite $R$-linear combination of monomials. That is, a finite linear combination of monomials with coefficients in the commutative ring $R$. 

\begin{defn}
   Let $f$ be a non-zero polynomial in $x_1, \ldots, x_n$. The sum of all monomial terms in $f$ of degree $k$ is called the \deff{homogeneous component of degree $k$ in $f$}.
\end{defn}

If $f$ is a nonzero polynomial having degree $d$, then we may write $f$ uniquely as $f_0 + \ldots f_d$ where $f_k$ is the homogeneous component of degree $k$ in $f$, for $0 \leq k \leq d$.

\subsection{Polynomial Rings over Fields}

We now consider the special case when the coefficient ring is itself a field, say $\F$. We define a norm $N$ on $\F[x]$ with $N(p(x)) = \deg p(x)$ and $N(0) = 0$. Recall from \Cref{prop:div_algo_fields} that the division algorithm holds. We restate this proposition for sake of completeness.

\begin{prop}[Division Algorithm]
    Let $\F$ be a field and let $f(x), g(x) \in \F[x]$ with $g(x) \neq 0$. Then, there are unique polynomials $q(x), r(x) \in \F[x]$ such that 
    \[
        f(x) = g(x)q(x) + r(x)
    \]
    with $r(x) = 0$ or $\deg r(x) < \deg g(x)$.
\end{prop}

Notice that $\deg r(x) = N(r(x))$ and $\deg g(x) = N(g(x))$. Thus, the above proposition states that $\F[x]$ is a Euclidean domain. An immediate corollary is the following.

\begin{cor} \label{cor:F[x]-is-UFD-and-PID}
    If $\F$ is a field, then $\F[x]$ is a principal ideal domain and a unique factorisation domain.
\end{cor}
\begin{proof}
    This is immediate since every Euclidean domain is a principal ideal domain and every principal ideal domain is a unique factorisation domain.
\end{proof}

\begin{prop}[Gauss' Lemma] \label{prop:gauss-lemma}
    Let $R$ be a unique factorisation domain and let $\F$ be its field of fractions. If $p(x) \in R[x]$ is reducible in $\F[x]$, then $p(x)$ is reducible in $R[x]$. More precisely, if $p(x) = A(x) B(x)$ for some non-constant polynomials $A(x), B(x) \in \F[x]$, then there exist nonzero elements $r,s \in \F$ such that $a(x) \vcentcolon= rA(x)$ and $b(x) \vcentcolon= sB(x)$ both lie in $R[x]$ and $p(x) = a(x) b(x)$ is a factorisation in $R[x]$. 
\end{prop}
\begin{proof}
    The coefficients of polynomials $A(x), B(x)$ are elements of $\F$ and hence quotients of elements in $R$. Multiplying throughout by a common denominator, we get $dp(x) = a^{\prime}(x) b^{\prime}(x)$, where $a^{\prime}(x), b^{\prime}(x) \in R[x]$, and $d$ is a nonzero element of $R$. If $d$ is a unit in $R$, then the proposition holds with $a(x) = d^{-1} a^{\prime}(x)$ and $b(x) = b^{\prime}(x)$. If not, we can write $d$ as a product of irreducibles in $R$ (since $R$ is a unique factorisation domain), say $d = p_1 \cdots p_n$. Since $p_1$ is irreducible and $R$ is a unique factorisation domain, $p_1$ is also prime by \Cref{prop:UFD-prime-iff-irreducible}. Hence, the ideal $\langle p_1 \rangle$ is prime. Now, by \Cref{prop:prime-ideals-maintained}, $p_1 R[x]$ is a prime ideal of $R[x]$ and $(R/p_1R)[x]$ is an integral domain. Reducing the equation $dp(x) = a(x)b(x)$ modulo $p_1$, we obtain $0 = \overline{a^{\prime}(x)} \,  \overline{b^{\prime}(x)}$, where the bar indicates the images of these polynomials in the quotient ring. Since $(R/p_1R)[x]$ is an integral domain, one of the factors, say $\overline{a^{\prime}(x)}$ must be zero. This means that all coefficients of $a^{\prime}(x)$ are divisible by $p_1$, so that $\frac{1}{p_1} a^{\prime}(x) \in R[x]$. Thus, from the equation $dp(x) = a(x) b(x)$, we can cancel a factor of $p_1$ from both sides while still having an equation in $R[x]$. Proceeding the same way with all remaining factors of $d$, we obtain $p(x) = a(x) b(x)$ as a factorisation in $R[x]$ with $a(x), b(x)$ being $\F$-multiples of $A(x), B(x)$ respectively.
\end{proof}

\begin{cor} \label{cor:gcd-1-implies-equivalence}
    Let $R$ be a unique factorisation domain, let $\F$ be its field of fractions, and let $p(x) \in R[x]$. If the greatest common divisor of coefficients of $p(x)$ is $1$, then $p(x)$ is irreducible in $R[x]$ if and only if $p(x)$ is irreducible in $\F[x]$. In particular, if $p(x)$ is a monic polynomial that is irreducible in $R[x]$, then $p(x)$ is irreducible in $\F[x]$.
\end{cor}
\begin{proof}
    By Gauss' Lemma (\Cref{prop:gauss-lemma}), if $p(x)$ is reducible in $\F[x]$, then it is irreducible in $R[x]$. Conversely, since the greatest common divisor of coefficients of $p(x)$ is $1$, we have that if $p(x)$ is reducible in $R[x]$, then $p(x) = a(x) b(x)$ where $a(x), b(x) \in R[x]$ are both non-constant. This same factorisation also shows that $p(x)$ is reducible in $\F[x]$, completing the proof. 
\end{proof}

\begin{theorem}
    $R$ is a unique factorisation domain if and only if $R[x]$ is a unique factorisation domain.
\end{theorem}
\begin{proof}
    If $R[x]$ is a unique factorisation domain, then $R$ is trivially a unique factorisation domain (since the factorisation of any element of $R$ in $R[x]$ must be a factorisation in $R$ itself, due to degree considerations). Suppose conversely that $R$ is a unique factorisation domain and let $\F$ be its field of fractions. Let $p(x)$ be a nonzero polynomial in $R[x]$. Let $d$ be the greatest common divisor of the coefficients of $p(x)$, so that $p(x) = dp^{\prime}(x)$ where the greatest common divisor of coefficients of $p^{\prime}(x)$ is $1$. Notice that $d$ is unique up to units in $R$ (which are also units in $R[x]$) and $d$ can be factored into irreducibles in $R$ (which are also irreducibles in $R[x]$). It suffices to show that $p^{\prime}(x)$ can be uniquely (up to units) factored into irreducibles in $R[x]$. We may hence assume that the greatest common divisor of coefficients of $p(x)$ is $1$ and that $p(x)$ is not a unit in $R[x]$, that is, $\deg p(x) > 0$. By \Cref{cor:F[x]-is-UFD-and-PID}, $\F[x]$ is a unique factorisation domain, and hence $p(x)$ can be factored uniquely as a finite product of irreducibles in $\F[x]$. Using Gauss' Lemma (\Cref{prop:gauss-lemma}) and \Cref{cor:gcd-1-implies-equivalence}, we can show that $p(x)$ can be written as a finite product of irreducibles in $R[x]$. We leave the proof of uniqueness as an exercise to the reader. It follows from the uniqueness of decomposition in $\F[x]$.
\end{proof}

\subsection{Irreducibility Criteria}

\begin{prop}
    Let $\F$ be a field and let $p(x) \in \F[x]$. Then $p(x)$ has a factor of degree one if and only if $p(x)$ has a root in $\F$, that is, there is an $\alpha \in \F$ with $p(\alpha) = 0$.
\end{prop}
\begin{proof}
    Since $\F$ is a field, if $p(x)$ has a factor of degree one, we may assume it to be monic, i.e, of the form $(x-\alpha)$ for some $\alpha \in \F$. Then clearly $p(\alpha) = 0$. Conversely, suppose that $p(\alpha) = 0$ for some $\alpha \in \F$. By the division algorithm in $\F[x]$ (\Cref{prop:div_algo_fields}), we may write
    \[
        p(x) = q(x) (x-\alpha) + r
    \]
    where $r$ is a constant (since $\deg r < \deg (x-\alpha)$). Since $p(\alpha) = 0$, $r$ must be zero, and hence $p(x)$ has $(x-\alpha)$ as a factor.
\end{proof}

\begin{prop}
    A polynomial of degree two or three over a field $\F$ is reducible in $\F[x]$ if and only if it has a root in $\F$.
\end{prop}
\begin{proof}
    We leave this as an exercise.
\end{proof}

\begin{prop} [Rational Root Theorem] \label{prop:rational-root-thm}
    Let $p(x) = a_nx^n + \ldots + a_0$ be a polynomial of degree $n$ in $\Z[x]$. If $\frac{r}{s} \in \Q$ is in lowest terms (i.e, $\gcd(r,s) = 1$), and $\frac{r}{s}$ is a root of $p(x)$, then $r \divides a_0$ and $s \divides a_n$.
\end{prop}
\begin{proof}
    We have
    \[
        p\left( \frac{r}{s} \right) = a_n \left( \frac{r}{s} \right)^n + \ldots + a_0 = 0 \implies a_n r^n + a_{n-1} r^{n-1}s + \ldots + a_0 s^n = 0
    \]
    Thus, we have $a_n r^n = s(-a_{n-1}r^{n-1} - \ldots - a_0 s^{n-1})$ so that $s \divides a_n r^n$. Since $\gcd(r,s) = 1$, it follows that $s \divides a_n$. The proof for $r \divides a_0$ follows along similar lines.
\end{proof}

\begin{cor}
    Let $p(x) \in \Z[x]$ be monic. If $p(d) \neq 0$ for all integers $d$ dividing the constant term of $p(x)$, then $p(x)$ has no root in $\Q$.
\end{cor}
\begin{proof}
    This follows trivially from \Cref{prop:rational-root-thm}.
\end{proof}

\begin{prop}
    Let $I$ be a proper ideal\footnotemark\ in the integral domain $R$. Let $p(x)$ be a non-constant monic polynomial in $R[x]$. If the image of $p(x)$ in $(R/I)[x]$ cannot be factored in $(R/I)[x]$ into two polynomials of smaller degree, then $p(x)$ is irreducible in $R[x]$.
\end{prop}
\footnotetext{That is, $I$ is an ideal of $R$ and $I \neq R$.}
\begin{proof}
    Suppose that $p(x)$ cannot be factored in $(R/I)[x]$ but is reducible in $R[x]$. Then, there exist non-constant polynomials $a(x), b(x) \in R[x]$ such that $p(x) = a(x) b(x)$. Moreover, $a(x), b(x)$ are both monic since $p(x)$ is monic. By \Cref{prop:prime-ideals-maintained}, reducing the coefficients modulo $I$ gives us a non-constant factorisation in $(R/I)[x]$, which is a contradiction.
\end{proof}

\begin{prop}[Eisenstein-Schönemann Criterion] \label{prop:ES-criterion}
    Let $P$ be a prime ideal of the integral domain $R$. Let $f(x) = x^n + a_{n-1} x^{n-1} + \ldots + a_0$ be a polynomial in $R[x]$ ($n \geq 1$). If $a_{n-1}, \ldots, a_0 \in P$ and $a_0 \notin P^2$, then $f(x)$ is irreducible in $R[x]$.
\end{prop}
\begin{proof}
    Suppose $f(x)$ were reducible in $R[x]$, say $f(x) = a(x) b(x)$ where $a(x), b(x) \in R[x]$ are non-constant polynomials. Reducing this equation modulo $P$, we obtain $x^n = \overline{a(x)} \, \overline{b(x)}$ in $(R/P)[x]$, where the bar indicates polynomials whose coefficients are reduced modulo $P$. Since $P$ is a prime ideal, $R/P$ is an integral domain, it follows that the constant terms of both $\overline{a(x)}$ and $\overline{b(x)}$ are $0$, that is, the constant terms of both $a(x)$ and $b(x)$ are elements of $P$. However, this implies that the constant term $a_0$ of $f(x)$ is an element of $P^2$, which is a contradiction.
\end{proof}

\Cref{prop:ES-criterion} is most frequently used in the case of $\Z[x]$, so we state this result explicitly as a corollary. 

\begin{cor}[Eisenstein-Schönemann Criterion for Integers] \label{cor:ES-criterion-for-Z}
    Let $p$ be a prime in $\Z$ and let $f(x) = x^n + a_{n-1}x^{n-1} + \ldots + a_0$ be a polynomial in $\Z[x]$, with $n \geq 1$. If $p \divides a_i$ for all $i \in \{0, \ldots, n-1\}$ but $p^2 \notdivides a_0$, then $f(x)$ is irreducible in both $\Z[x]$ and $\Q[x]$.
\end{cor}
\begin{proof}
    Trivial.
\end{proof}